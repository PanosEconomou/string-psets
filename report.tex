\documentclass{homework}

\title{String Theory Final}
\author{Panos Oikonomou}

\begin{document}
\maketitle
\section*{Orbifolds}
How do we make string theory have physics that look like the standard model? We have been using two tools to achieve this: ``editing'' the degrees of freedom of the embedding, and modifying the spacetime the string is embedded in. For example adding fermionic degrees of freedom allowed us to find spacetime Fermions in certain settings. But what really unlocks the door to realistic looking physics is changing the target space in various fun ways.

This is where orbifolds come in. Studying the toolkit of orbifolds will allow us to write down theories in nontrivial target space configurations that lead to interesting physics. In these notes, we begin with an intuitive description of orbifolds through the example of a single compact boson and its $\mathbb{Z}_2$ orbifold, then highlight some generalizations, and finally apply this to the heterotic string compactification that looks a lot like a standard model unification.

\tableofcontents

\section{Orbifolding the Compact Boson}
To study the toolkit of orbifolds we first play with the familiar case of the compact Boson. We can think of a compact Bosonic field $X : \Sigma \to S^1$ as a way to ``wrap'' the 2-dimensional Riemann surface $\Sigma$ on a circle. This interpretation allows us to steal a core mathematical idea to physics.
\subsection{Mathematical Orbifolds}
Let $G$ be some finite group with an action on a smooth manifold $M$, then the orbifold $M/G$ is the topological space of $G$ equivalence classes on $M$. In other words an orbifold is the shape we obtain by ``folding'' all the points on a manifold that are related by a symmetry on top of each other.

Back to the example of a circle, we have a $\mathbb{Z}_2$ action that maps $\theta \mapsto -\theta$, where $\theta$ is the angle relative to the fixed point of the action. We can then construct the familiar orbifold of the circle which ends up looking like a line segment. The real physics question is, \textit{can we use the field theory that folds $\Sigma$ to $S^1$ to build a theory that folds it to its orbifold instead}? 
\subsection{Untwisted Sector}
To answer this we can cast this orbifold procedure into physics language. We notice that the $\mathbb{Z}_2$ action we described is a global symmetry of the compact free Boson theory, while on the orbifold it is as if this symmetry has become local because the point $X(\sigma, \tau)$ and $-X(\sigma, \tau)$ are the same. So this idea of using the theory of the circle to obtain a theory on the orbifold is identical to gauging a global symmetry. Not only that, but we have already seen examples of this in places like the orientifold construction and the GSO projection. So let's see it in practice.

When we say we want to ``find the theory'' what we really mean is to find some way to describe the Hilbert space. We can start by asking what states of the compact free boson remain in the theory defined on the orbifold. Such states are obtained by acting with the mode operators $\alpha_n$ of a Boson with periodic boundary conditions such that under the transformation $a_n\mapsto -a_n$ they remain invariant.

\subsection{Twisted Sectors}
\subsection{Cheating via Modularity}
\subsection{Generalizing}
% talk about how we secretly used the symmetries of $X$ to find out that we can orbifold the circle, and when we introduce things Fermions we have to find the full symmetry group. 

\section{Heterotic String Orbifolds}
The $E_8 \times E_8$ heterotic string has 10 free bosons $X$ with periodic boundary conditions, 32 left moving positive chirality Majorana spinors $\lambda$ half of them with periodic and the other half with antiperiodic boundary conditions, and 10 right moving positive chirality Majorana Fermions and either periodic or antiperiodic boundary conditions.
\subsection{Obstructions to gauging}
\subsection{The $\mathbb{Z}_3$ Orbifold}
\subsection{Spacetime Supersymmetry}
\subsection{$E_6$ Spectrum and Grand Unification}
\end{document}
