\documentclass{homework}

\title{String Theory Final}
\author{Panos Oikonomou}

\begin{document}
\maketitle
\section*{Orbifolds}
How do we make string theory have physics that look like the standard model? We have been using two tools to achieve this: ``editing'' the degrees of freedom of the embedding, and modifying the spacetime the string is embedded in. For example adding fermionic degrees of freedom allowed us to find spacetime Fermions in certain settings. But what really unlocks the door to realistic looking physics is changing the target space in various fun ways.

This is where orbifolds come in. Studying the toolkit of orbifolds will allow us to write down theories in nontrivial target space configurations that lead to interesting physics. In these notes, we begin with an intuitive description of orbifolds through the example of a single compact boson and its $\mathbb{Z}_2$ orbifold, then highlight some generalizations, and finally apply this to the heterotic string compactification that looks a lot like a standard model unification.

\tableofcontents

\section{Orbifolding the Compact Boson}
To study the toolkit of orbifolds we first play with the familiar case of the compact Boson. We can think of a compact Bosonic field $X : \Sigma \to S^1$ as a way to ``wrap'' the 2-dimensional Riemann surface $\Sigma$ on a circle. This interpretation allows us to steal a core mathematical idea to physics.
\subsection{Mathematical Orbifolds}
Let $G$ be some finite group with an action on a smooth manifold $M$, then the orbifold $M/G$ is the topological space of $G$ equivalence classes on $M$. In other words an orbifold is the shape we obtain by ``folding'' all the points on a manifold that are related by a symmetry on top of each other.

Back to the example of a circle, we have a $\mathbb{Z}_2$ action that maps $\theta \mapsto -\theta$, where $\theta$ is the angle relative to the fixed point of the action. We can then construct the familiar orbifold of the circle which ends up looking like a line. The real physics question is, \textit{can we use the field theory that folds $\Sigma$ to $S^1$ to build a theory that folds it to its orbifold instead}? 
\subsection{Untwisted Sector}
To answer this question 
\subsection{Twisted Sectors}
\subsection{Cheating via Modularity}

\section{General Lessons from the Compact Boson}
% talk about how we sectretly used the symmetries of $X$ to find out that we can orbifold the circle, and when we introduce things Fermions we have to find the full symmetry group. 

\section{Heterotic String Orbifolds}
The $E_8 \times E_8$ heterotic string has 10 free bosons $X$ with periodic boundary conditions, 32 left moving positive chirality Majorana spinors $\lambda$ half of them with periodic and the other half with antiperiodic boundary conditions, and 10 right moving positive chirality Majorana Fermions and either periodic or antiperiodic boundary conditions.
\end{document}
