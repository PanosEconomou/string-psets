\documentclass{homework}
\usepackage{svg}
\usepackage{multirow}
\usepackage{booktabs}

\title{String Theory Final}
\author{Panos Oikonomou}

\begin{document}
\maketitle
\section*{Orbifolds}
How do we make string theory have physics that look like the standard model? We have been using two tools to achieve this: ``editing'' the degrees of freedom of the embedding, and modifying the spacetime the string is embedded in. For example adding fermionic degrees of freedom allowed us to find spacetime fermions in certain settings. But what really unlocks the door to realistic looking physics is changing the target space in various fun ways.

This is where orbifolds come in. Studying the toolkit of orbifolds will allow us to write down theories in nontrivial target space configurations that lead to interesting physics. In these notes, we begin with an intuitive description of orbifolds through the example of a single compact boson and its $\mathbb{Z}_2$ orbifold, then highlight some generalizations, and finally apply this to the heterotic string compactification that contains something a lot like a standard model unification.

\tableofcontents

\section{Orbifolding the Compact boson}
% To study the toolkit of orbifolds we first play with the familiar case of the compact boson. This is not meant to be an exhaustive review, rather a familiar example which highlights things that we expect on the heterotic string orbifold which have a murkier geometric interpretation. 
We can think of a compact bosonic field $X : \Sigma \to S^1$ as a way to ``wrap'' the 2-dimensional Riemann surface $\Sigma$ on a circle. This interpretation allows us to steal a core mathematical idea to physics.
\subsection{Mathematical Orbifolds}
Let $G$ be some finite group with an action on a smooth manifold $M$, then the orbifold $M/G$ is the topological space of $G$ equivalence classes on $M$. In other words an orbifold is the shape we obtain by ``folding'' all the points on a manifold that are related by a symmetry on top of each other.

Back to the example of a circle, a $\mathbb{Z}_2$ action maps $X \mapsto -X$, where $X$ is the angle relative to the fixed point of the action as shown in Fig.~\ref{fig:circle-orbifold}. We can then construct the familiar orbifold of the circle which ends up looking like a line segment. The physics question is, \textit{can we use the field theory that folds $\Sigma$ to $S^1$ to build a theory that folds $\Sigma$ to the orbifold instead}?
\begin{figure}
	\centering
	\includesvg[width=0.7\textwidth]{media/circle_orbifold}
	\caption{The $\mathbb{Z}_2$ orbifold of the circle under the $X\mapsto - X$ map. The fixed point is shown in green.}\label{fig:circle-orbifold}
\end{figure}

\subsection{Untwisted Sector}
To answer this we can cast this orbifold procedure into physics language. The $\mathbb{Z}_2$ action is a global symmetry of the compact free boson theory, while on the orbifold it is as if this symmetry has become local because the point $X(\sigma, \tau)$ and $-X(\sigma, \tau)$ are the same. So this idea of using the theory of the circle to obtain a theory on the orbifold is identical to gauging a global symmetry. We have already seen examples of this in places like the orientifold construction and the GSO projection.

We can start by asking what states of the compact free boson remain in the theory defined on the orbifold. Such states are obtained by acting with the mode operators $\alpha_n$ of a boson with periodic boundary conditions such that under the transformation $r:\alpha_n\mapsto -\alpha_n$ they remain invariant. This is simply the states with an even number of generators. In fact we can calculate the torus partition function by projecting out the states with a negative $r$ eigenvalue to obtain (as we already have in a homework)
\begin{align*}
	\hat Z 
	&= \text{Tr}_{\mathbb{H}} \frac{1 + r}{2} q^{L_0 - \frac{1}{24}} \bar q^{\tilde L_0 - \frac{1}{24}}
	= \frac{(q \bar q)^{\frac{1}{24}}}{2} \prod_{n \in \mathbb{N}}\abs{1 - q^n}^{-2} + \frac{(q \bar q)^{\frac{1}{24}}}{2} \prod_{n\in \mathbb{N}} \abs{1 + q^{n}}^{-2}\\
	&= \frac{1}{2\sqrt{4\pi^2 \alpha'\text{Im\,}\tau} \abs{\eta(\tau)}^2} + \abs{\frac{\eta(\tau)}{\theta_{10}(\tau)}}.
\end{align*}
Here we recognize the first term as (half) the partition function of the free boson, while the second is the term that accounts for subtracting the degrees of freedom that are not invariant under $r$.
\subsection{Twisted Sectors}
The partition function we have calculated is not modular invariant because $\theta_{10}(-1/\tau) \propto \theta_{01}(\tau)$. This indicates that the theory must have more states than the ones that we have counted and once we account for them, we will get the extra terms in the torus partition function that will restore modularity. This makes sense because we really only considered ways to wrap $\Sigma$ around a circle such that this was invariant under the orbifold. 

If instead the free boson had the property $X(\sigma + \pi,\tau) = - X(\sigma,\tau)$ while this would not lead to a valid ``wrapping'' around a circle, in the orbifold it's still fine! So let's include states that have these boundary conditions. States such that their holonomy is up to the group action we are gauging by are said to belong to the \textit{twisted sectors} of the theory since we wouldn't find them in the local Hilbert space.

Thankfully in the homework we have also calculated the correct phase that we need to shift $L_0$ by in the twisted sector by calculating the vacuum energy contribution of the twisted boundary conditions. So the trace of the invariant part of the twisted sector is given by
\begin{align*}
	\hat Z_\sigma &=
	\text{Tr}_{\mathbb{H}_r} \frac{1+r}{2} q^{L_0 + \frac{1}{48}} \bar q^{\tilde L_0 + \frac{1}{48}} 
	= (q\bar q)^{\frac{1}{48}} \prod_{n \in \mathbb{N}} \abs{1-q^{n-\frac{1}{2}}}^{-2} (q\bar q)^{\frac{1}{48}} \prod_{n \in \mathbb{N}} \abs{1+q^{n-\frac{1}{2}}}^{-2}\\
	&= \abs{\frac{\eta(\tau)}{\theta_{10}(\tau)}} + \abs{\frac{\eta(\tau)}{\theta_{00}(\tau)}}.
\end{align*} 
Adding these two terms into our partition function from above fixes our modular invariance issue since the first term of the twisted partition function transforms into the second term of the untwisted one, while the last term is modular invariant by itself.
\subsection{Generalizing}
We are by no means done with exploring the $\mathbb{Z}_2$ orbifold of the free boson, but we have encountered enough cool properties to motivate the general construction that we will adopt on the Heterotic string in the next section. 

The first important observation is that the procedure of orbifolding a string theory was to examine how the target space transformations we want to orbifold by form a set of global symmetries for the underlying CFT and gauging that. There might not always be a unique way to interpret which global symmetries correspond to the target space transformations we want to orbifold by.

We used modularity to observe that twisted sectors appear. When we gauge in a general QFT we not only have to path integrate over all configurations of the connection one form, but also have to sum over all principal bundles. This has a physical effect that we see states with different masses than in the previous theory which come from the shift of the vacuum energy in the twisted sector.
\section{Heterotic String Orbifolds}
For more interesting physics let's examine what happens when we orbifold the heterotic string by a discrete Abelian group. The setup is as follows. We compactify the 10 dimensional heterotic string on a 6 dimensional torus $T^6$ formed by a lattice $\Lambda$ that we will describe in a bit. Then we take some symmetries of the torus and form an Abelian group $G$ and orbifold by them (Technically the toroidal compactification is in itself an orbifold, but we can start directly from there). Therefore the final target space looks like
\begin{equation*}
	\mathbb{R}^{3,1}\times (T^6/G)
\end{equation*}
Assume that we have found a rotation of the torus $r$ with index $k \in \mathbb{N}$, i.e.\ such that $r^k = 1$ then we can find out how such a rotation can be interpreted as a symmetry of the fields in the heterotic string and effectively orbifold by $\mathbb{Z}_k \coloneqq \langle r^k\rangle$. Such a rotation will be part of $SO(6)$ and we can always change our coordinates such that the rotation is along the generators of a $U(1)^3 < SO(6)$. Let's call these generators $J_a$ for $a\in\{1,2,3\}$ and the corresponding angles $\theta_a$, which implies that $r=e^{i\theta \cdot J}$ when it acts on the bosons.

In these coordinates we have effectively split the last 6 toroidal coordinates into three pairs where a $J_a$ rotates the $a^{\text{th}}$ pair. We can define $Z^a = \frac{1}{\sqrt{2}} \left(X^{a_1} + i X^{a_2}\right)$, where $a_1, a_2$ are the indices corresponding to the coordinates that $J_a$ is rotating. Therefore we can write the embedding coordinates as $X = (X^0,\cdots, X^3, Z^1, \bar Z^1, \cdots, Z^3, \bar Z^3)$. The reason for doing this is that now they transform as complex bosons under $r$ 
\begin{align*}
	r \cdot Z^a = e^{i\theta_a} Z^a.
\end{align*}
This theory doesn't only have bosons though. Thanks to supersymmetry we know that the antichiral fermions transform the same way after we define the Dirac fermions $\tilde \phi^a = \frac{1}{\sqrt{2}}\left( \tilde \psi^{a_1} + i \tilde \psi^{a_2} \right)$
\begin{align*}
	r \cdot \tilde \phi^a = e^{i\theta_a} \tilde \phi^a.
\end{align*}
Yet, this transformation doesn't fix how the supercurrents $\lambda$ transform. We know that thy must transform under $SO(16)\times SO(16)$ since they are the currents that generate it (the remaining currents that form $E_8\times E_8$ are the fermionic vertex operators). So it seems that we have freedom to choose a gauge transformation for the $\lambda$'s. Not that much freedom thought. If we assume that the identity transformation in the target space comes with the identity transformation of the $\lambda$'s, then it should be the case that the action of $r$ on the supercurrents still satisfies $r^k = 1$. 

Choosing the $\lambda$ appropriately, a general transformation of $SO(16)\times SO(16)$ can be written in terms of generators of $U(1)^{16} < SO(16)\times SO(16)$ so if we call these generators $S_a$ for $a \in {1,\cdots, 16}$ and the corresponding angles $\beta_a$ we have that $r = e^{i\theta\cdot J + i\beta \cdot S}$. We can do a similar trick and define $\mu^a = \frac{1}{\sqrt{2}}\left( \lambda^{a_1} + i \lambda^{a_2} \right)$ such that $r$ acts as
\begin{align*}
	r \cdot \mu^a = e^{i\beta_a}\mu^a.
\end{align*}
A convenient choice of angle is $\beta^a = \theta^a$ for $a\in \{1,\cdots 4\}$ and zero otherwise. Now we know how to interpret the target space orbifold in terms of a symmetry of the fields in our string theory. 

As a final remark the finite index of the rotation $r$ implies that $\frac{k\theta_a}{2\pi}, \frac{k\beta_a}{2\pi} \in \mathbb{Z}$ as a further constraint.
\subsection{Obstructions to gauging}
One key issue that wasn't evident in the free boson orbifold is that in general we might not be able to construct a consistent string theory for any orbifold. Phases between the terms in the torus partition function can potentially break modularity. This happens when we can't assign a phase to the path integral over a sector that is independent of large gauge transformations. Such an anomaly is called a gravitational anomaly and we must ensure that whatever we try to gauge doesn't lead to one. Thankfully, it can be shown that, if we try to orbifold an Abelian group, the only kind of anomaly that can break modularity is gravitational.

The partition function for any well defined string theory is manifestly modular invariant in the untwisted sector. Therefore, if we see any anomaly it will appear in the partition function contributions from the twisted sectors.

In the twisted sector the fields have boundary conditions augmented by the symmetry action. In our case here is a summary of how that looks like for some natural number $0<n<k$
\begin{align*}
	Z^a(\sigma + \pi) = e^{i\theta_a}Z^a(\sigma) && \tilde \phi^a(\sigma + \pi) = e^{i\theta_a + i\nu} \tilde \phi^a(\sigma) && \mu^a(\sigma + \pi) = e^{i\beta_a + i\nu} \mu^a(\sigma),
\end{align*}
where $\nu$ is an angle that defines the $R$ or $NS$ boundary condition of the corresponding field. The reason why this is relevant is that the modes vacuum energy will be shifted for each field and it might happen that the leading power of $q$ in the partition function is no longer invariant under $\tau \mapsto \tau + 1$. To show this, we know that the partition function is given by
\begin{align*}
	\text{tr\,} q^{L_0- \frac{c}{24}} \bar q^{\tilde L_0 - \frac{c}{24}}\xRightarrow{\tau \mapsto \tau+1} \text{tr\,} e^{2\pi i (L_0 - \tilde L_0)}q^{L_0 - \frac{c}{24}} \bar q^{\tilde L_0- \frac{c}{24}}.
\end{align*}
Therefore what we would like is that first factor to be 1 so that the partition function is invariant, which means that $L_0 - \tilde L_0$ should be an integer over the spectrum.

First in the sector where all the fermions are in the Ramond sector $(\nu =0)$ we use the result we derived in the homework to calculate the level mismatch between the left and right movers. In particular we know that the change in the zero point energy for a complex boson is given by $\Delta E_b = -\frac{\theta}{4\pi}\left( \frac{\theta}{2\pi} - 1 \right)$, and minus that for the fermion. Then we can find that
\begin{align*}
	L_0 - \tilde L_0 \  \text{mod\,}1
	&= - \sum_{a=1}^{3} (N_Z^a - N_{\bar Z}^{a} + \tilde N_Z^a - \tilde N_{\bar Z}^{a} + \tilde N_{\phi}^a - \tilde N_{\bar \phi}^a)\frac{\theta_a}{2\pi} - \sum_{a=1}^{16} (N_{\mu}^a - N_{\bar \mu}^a)\frac{\beta_a}{2\pi}\\
	&\ \ \ \, + \sum_{a=1}^{3} \frac{\theta_a}{4\pi}\left( \frac{\theta_a}{2\pi} - 1 \right) - \sum_{a=1}^{16}\frac{\beta_a}{4\pi}\left( \frac{\beta_a}{2\pi} - 1 \right),
\end{align*}
where $N^a_\phi$ is the number operator for the modes of field $\phi^a$. Notice that the top term is an integer multiple of $1/k$, while the bottom is an integer multiple of $1/2k^2$, therefore if we were to simply include the twisted sectors as they are in the partition function, then it would not be modular invariant. In fact, there is a chance that we pick phases such that the bottom contribution from the vacuum energy (i.e. the conformal weight of the vacuum) never cancels. Therefore we must impose the condition
\begin{align*}
	\sum_{a=1}^{3} \frac{\theta_a}{4\pi}\left( \frac{\theta_a}{2\pi} - 1 \right) - \sum_{a=1}^{16}\frac{\beta_a}{4\pi}\left( \frac{\beta_a}{2\pi} - 1 \right)\  \text{mod\,}1 = \frac{m}{k},
\end{align*}
for some integer $m$ at least, in order to even have a chance that these two would cancel out. This is not enough however, because there is no reason why the oscillators would cancel this by an integer. Thankfully, we are not summing over all possible states, only the ones that are invariant under $\mathbb{Z}_k$. In fact, if we calculate the generators $J_a$ and $S_a$ using the corresponding creation and annihilation operators in the twisted sectors we find that the action of $r$ is exactly the oscillator part in $L_0 - \tilde L_0$, so imposing invariance implies that the oscillator part should cancel. Doing a similar analysis in the sectors where one set of fermions is $NS$ (i.e. $\nu = \pi$) and the rest $R$ we obtain the additional conditions
\begin{align*}
\sum_{a=1}^3 \theta_a, \sum_{a=1}^8 \beta_a, \sum_{a=9}^{16} \beta_a \in 4\pi \mathbb{Z}. 
\end{align*} 
\subsection[The Z3 Orbifold]{The $\mathbb{Z}_3$ Orbifold}
We finally have enough to describe the orbifold we want to take without running into pesky anomaly issues. The orbifold in this case is $\mathbb{Z}_3$ on a toroidal compactification of the Heterotic string. Namely, we will construct the torus in the $Z^a$ coordinates by the identifications $Z^a \sim Z^a + R^a$ and $Z^a \sim Z^a + \omega R^a$, where $\omega$ is the 3rd root of unity, i.e. $\omega^3 = 1$, and $R^a$ are some complex numbers defining the lattice spacing. This compactification is the torus $T^6$. The $\mathbb{Z}_3$ that we will find is generated by the rotation $r$ that acts on the coordinates as follows
\begin{align*}
r \cdot Z^1 = \alpha Z^1 && r \cdot Z^2 = \alpha Z^2 && r \cdot Z^3 = \alpha^{-2} Z^3.
\end{align*} 
Or in the notation we had before, $\theta = \left( \frac{2\pi}{3}, \frac{2\pi}{3}, -\frac{4\pi}{3} \right)$, while on the right moving supercurrents it acts the same way, with it permuting only the first three by $\theta$. This satisfies the constraints we found above, therefore we know that trying to orbifold by this group action will yield a modular invariant theory, which is particularly nice. There are other choices we could make for how the angles $\beta$ act, and these would correspond to coupling the theory to different singular flat connections, but we won't examine those. 

The whole point of doing this is to get some cool physics out of this here is what we will show in the next couple of sections. First we will calculate the massless spectrum of the theory carefully and highlight which representations survive in from the untwisted and twisted sectors. Then we will also justify why this orbifold has an unbroken $N=1$ supersymmetry which is pretty cool from a particle phenomenology standpoint since GUT people believe that this might be the case based on studying the self energy of standard model particles. For our purposes though, it is still cool because we get chiral spacetime fermions for free!
\subsection{Low Energy Spectrum}
Notice that the $\mathbb{Z}_3$ is chosen in a special way so as to act only on the first 3 generators $\mu$ of the gauge currents, so it would be convenient to decompose our states in representations of the largest subgroup that is invariant under that $\mathbb{Z}_3$. In fact, we can work out how $e^{i\beta \cdot S}$ is in the center of a $SU(3) < E_8\times E_8$, so what we can do is find the largest subgroup that contains the remaining $E_8$ roots. This turns out to be $E_6$ as shown in Fig.~\ref{fig:dynkin} using some Dynkin diagrams. So if we want to classify our massless representations in terms of eigenstates of $r$, we should decompose them under $SU(3)\times E_6\times E_8$.
\begin{figure}
    \centering
    \includesvg[width=0.7\textwidth]{media/E8-dynkin}
    \caption{$E_8$ Dynkin diagram showing how we can find an $SU(3)\times E_6$ subgroup. The boxed areas contain roots with the right lattices to generate the two factors.}
    \label{fig:dynkin}
\end{figure}
% The fact that we found $E_6$ is exciting on its own merit since at the next section we will see how studying the kind of $E_6$ representations that survive the orbifold will provide a point of contact with the $E_6$ grand unified theory.
To find the spectrum we split the physical massless states in representations of $SU(3) \times E_6\times E_8$ and label them by eigenvalues of $r$ on the right and left moving part of the Hilbert space separately. Then we put left and right together and find what states in the untwisted sector remain invariant under $\mathbb{Z}_3$ (and GSO). Then we will repeat for the twisted spaces.

\subsubsection{Untwisted Sector}
\textbf{Chiral Gauge.} The decomposition of the $E_8\times E_8$ adjoint representation $\mathbf{248}\oplus \mathbf{248}$ under $SU(3)\times E_6\times E_8$, is given in Polchinski as
\begin{align*}
\mathbf{248} \cong (\mathbf{8} \otimes \mathbf{1}) \oplus (\mathbf{1}\otimes \mathbf{78}) \oplus (\mathbf{3}\otimes \mathbf{27}) \oplus (\overline{\mathbf{3}} \otimes \overline{\mathbf{27}}),
\end{align*} 
where in notation $a\otimes b$, $a$ is a representation of $SU(3)$ and $b$ is a representation of $E_6$. It is worth noting that $\mathbf{78}$ is the adjoint representation of $E_6$, $\mathbf{8}$ the adjoint representation of $SU(3)$, and $\mathbf{3}$ a fundamental representation of $SU(3)$. Since $r$ is an element of the center of $SU(3)$ we only need to find out how it acts on $\mathbf{1}, \mathbf{3}, \overline{\mathbf{3}}, \mathbf{8}$.

Consider any eigenstate $\ket{v}$ of the $U(1)^{16}$ generators $S_a$ that we have used to label the center of $SO(16)\times SO(16)$ and define the eigenvalues $S_a \ket{v} = s_i\ket{v}$. Then, we know $e^{i\beta \cdot S} \ket{v} = e^{i\beta\cdot s} \ket{v}$. We can use this to find out how $r$ acts on each representation of the gauge group. The center of $SU(3)$ is $Z_{SU(3)} =\{1, \omega, \omega^2\}$, so in the fundamental representation $\mathbf{3}$, we have that all the states have the same eigenvalues $s_1 = 0, s_2 = 1/3, s_3 = 2/3$ which implies that $r \ket{v} = \omega\ket{v}$. For the anti-fundamental $\overline{\mathbf{3}}$ the order of generators is permuted so we get $r \ket{v} = \omega^{-1}\ket{v} = \omega^2 \ket{v}$. Finally in the adjoint representation $\mathbf{8}$ the center acts as identity, so we get $r \ket{v} = \ket{v}$. Also notice that $\mathbf{334} = (\mathbf{8}\otimes \mathbf{1} \otimes \mathbf{1})\oplus(\mathbf{1}\otimes \mathbf{78}\otimes \mathbf{1})\oplus(\mathbf{1}\otimes\mathbf{1}\otimes \mathbf{248})$ is the adjoint representation of $SU(3)\times E_6\times E_8$.

\textbf{Chiral bosons.} There are more physical left moving states than the ones created using modes of the $\lambda$s. In particular we have found that any state of the form $\ket{v} = \alpha_{-1}^{\mu} \ket{0}$ satisfies $(L_0 - 1)\ket{v} = 0$. So it would be nice to see how these decompose into eigenstates of $r$. Here though, things are easier since $r$ doesn't act when $\mu \in \{2,3\}$, so such states satisfy $r \ket{v} = \ket{v}$. On the other hand, modes of $Z^a$ look like $z^a\coloneqq\frac{1}{\sqrt{2}}\left(\alpha_{k}^{a_1} + i \alpha_k^{a_2}\right)$. Since $r Z^a = \omega Z^a$ we find that states of the form $\ket{v} = z^a_{-1} \ket{0}$ mush satisfy $r \ket{v} = \omega \ket{v}$. Similarly, states of the form $\ket{v} = \bar z^a_{-1} \ket{0}$ mush satisfy $r \ket{v} = \omega^2 \ket{v}$. A summmary of the decomposition into $r$ eigenvalues is in Table~\ref{tab:untwisted}.

\textbf{Antichiral NS.} In the right moving sector, the massless heterotic string excitations are $\mathbf{8}_c$ in the Ramond sector, and $\mathbf{8}_v$ in the Neveau-Schwartz sector as fundamental $SO(8)$ representations. These the vector representation in the NS sector is given by state of the form $\ket{v} = \tilde \psi_{-\frac{1}{2}}^{\mu} \ket{0}_{NS}$ which we can classify using the argument above in the trivial $r$ eigenspace for $\mu \in \{2,3\}$, in the $\omega$ eigenspace for states of the form $\ket{v} = \tilde \phi^a_{-\frac{1}{2}} \ket{0}_{NS}$, and the $\omega^2$ eigenspace for the states in the conjugate representations.

\textbf{Antichiral R.} For the states in the spinor representation $\mathbf{8}_c$, we can classify them by splitting $SO(8)$ under $SU(3) < SO(6) < SO(8)$. In fact the $\mathbb{Z}_3$ we have foud only permutes the $\tilde \phi$, but leaves $\tilde \psi^1 + i\tilde \psi^2$ invariant. Therefore we know that under the chosen $SU(3)$ only  
\begin{align*}
	\mathbf{8}_c = \mathbf{3} \oplus \mathbf{1} \oplus \overline{\mathbf{1}} \oplus \overline{\mathbf{3}},
\end{align*}
where $\mathbf{1}, \overline{\mathbf{1}}$ are the two singlets with an eigevalue $1$ under $r$, that have either all positive or all negative spins under the $\tilde \phi^a$, and $\mathbf{3}, \overline{\mathbf{3}}$ are the two fundamental representations of $SU(3)$, here with 1 or 2 positive spins under $\tilde \phi^a$ respectively and with $\omega$ and $\omega^2$ eigenvalues under $r$. This can be shown via dimension counting, since the only other representation that firs here is $\mathbf{6}$ but if this was the case, we would be able to change the spin of $\tilde \psi^1 + i\tilde \psi^2$, which our embedding of $SU(3)$ is supposed to leave invariant. Everything is summed up in Table~\ref{tab:untwisted}.

\begin{table}
    \begin{center}
    \begin{tabular}{ccc}
    &\multicolumn{2}{c}{\textbf{Representation} or \textbf{Generators}}\\
    \bottomrule\midrule
    $r$ & {Left Massless } & Right Massless\\
    \hline
    $1$ 		& $\mathbf{334}$
				, $\alpha^{\mu}_{-1} \ket{0}$ 
				& $\mathbf{1}$ 
				, $\overline{\mathbf{1}}$
				, $\psi^{\mu}_{-\frac{1}{2}} \ket{0}_{NS}$ \\
    $\omega$    & $\mathbf{3} \otimes \mathbf{27} \otimes \mathbf{1}$
    			, $z^{a}_{-1} \ket{0}$ & $\mathbf{3}$
				, $\phi^{a}_{-\frac{1}{2}} \ket{0}_{NS}$ \\
    $\omega^2$  & $\overline{\mathbf{3}}	\otimes \overline{\mathbf{27}}	\otimes \mathbf{1}$
    			, $\bar z^{a}_{-1} \ket{0}$ 
				& $\overline{\mathbf{3}}$
				, $\bar \phi^{a}_{-\frac{1}{2}} \ket{0}_{NS}$ \\
    \hline
    \end{tabular}
    \caption{Collection of massless states in the left moving and right moving sectors written as either a basis, or a representation of $SO(3)\times E_6\times E_8$ for the left movers or $SO(8)$ for the right as decribed in the text.\label{tab:untwisted}}
    \end{center}
\end{table}

\subsubsection{Twisted Sectors}
In the twisted sector the complex fields $\Phi$ have periodicity $\Phi(\sigma + \pi) = r \cdot \Phi(\sigma)$. We know how to treat the individual bosonic or fermionic degrees of freedom with twisted boundary conditions, by shifting the commutation relations of the mode algebra. 

\textbf{Antichiral NS.} Let's calculate the vacuum energy for the right moving Neveau-Schwartz states to find the physical state condition in this sector. A complex fermion with boundary conditions $\tilde \phi(\sigma + \pi) = e^{i\theta} \tilde \phi(\sigma)$ has a vacuum energy shift of $\frac{\theta}{4\pi}\left( \frac{\theta}{2\pi} - 1 \right)$. Three fermions have periodicity $\frac{2\pi}{3} + \pi$ which implies that the shift of the vaccum energy is $-\frac{5}{72}$, while one has $\pi$ for a shift of $-\frac{1}{8}$. For the bosons we repeat this and get $\frac{1}{9}$ and for each. In total, the vacuum energy shift in the twisted sector is $3\frac{1}{9} - 3\frac{5}{72} - \frac{1}{8} = 0$. So in the twisted sector $a_{NS}^r = 0$ therefore the only massless state in the orbifold is the vacuum, which has to be invariant under $r$.

\textbf{Antichiral R.} In the right moving Ramond sector in the twisted space by $r$ the $\tilde \psi^\mu$ for $\mu \in \{2,3\}$ are unchanged, while the complex ones $\tilde \phi^a$ have modes shifted by $\frac{2\theta^a}{2\pi} = \frac{2}{3}$. This means that now, instead of having $2^4$ zero modes, we only have $2$, i.e. the ones from the spacetime fermions (the remaining ones are shifted). There is no reason though that these states are massless or even physical. To find this we need to calculate the zero point energy. We have as many fermionic or bosonic modes, that are shifted in the exact same way. Therefore the physical state condition remains unchanged, meaning that if any of the two vacua survive GSO, they would be massless states in the orbifold.

What we don't know, however, is how to impose GSO on the twisted sector properly. For GSO we had to define fermion parity $F = s^1 + s^2 + s^3 + s^4$ and keep the states $\ket{v}$ such that $(-1)^F \ket{v} = \ket{v}$, where $s^i$ are the eigenvalues of the $U(1)^4$ generators at the center of $SO(8)$. 

The issue here, is that in the twisted sector we don't know how the last three of them act since the bosons have additional twisted boundary conditions. No matter though, we can quickly fix that by looking at the bosonised vertex operator. More precisely an antichiral Dirac fermion with periodicity $\tilde \phi(\sigma + \pi) = e^{i\theta} \tilde \phi(\sigma)$ has vertex operator $e^{-i \frac{\pi - \theta}{2\pi} S}$, where $S$ is the corresponding $U(1)$ generator, where in our case $\frac{\pi - \theta_a}{2\pi} = \frac{1}{9}$. So the two ground states have weights $\left(\pm \frac{1}{2}, -\frac{1}{6}, -\frac{1}{6}, - \frac{1}{6}\right)$, and only the one with $s^1 = \frac{1}{2}$ survives GSO. Since this is the unique vacuum in this sector it has to transform trivially under $r$.

\textbf{Chiral.} To find the the physical state condition we calculate the zero-point energies. The shifts in the fermion the ground energy as a function of the angle that defines the boundary conditions are
\begin{align*}
    \textbf{NS: }  \pi & : -\frac{1}{8} & \pi \pm \frac{2\pi}{3} &: -\frac{5}{72} &
    \textbf{R: }  0 & : 0   & \pm \frac{2\pi}{3} &: - \frac{1}{9}.
\end{align*}
In the $(R,NS)$ the zero point energy shift vanishes, therefore massless states are vacuum states. Since the modes for $\mu^{a}$ have an extra twist they have higher energy so they don't form vacuum states, the remaining $10$ complex fermions form the fundamental spinor representations $\mathbf{16} \oplus \overline{\mathbf{16}}$ of $SO(10)$. Using the same argument about GSO as above we can see that only the $\overline{\mathbf{16}}$ survives. 

In the $(NS,NS)$ sector the zero point energy shift is $-\frac{1}{2}$ so we can generate, but each complex fermion mode contributes either $\frac{1}{6}$ or $\frac{1}{2}$ to the level. So we can find a singlet $\mathbf{1}$ formed by $\mu^1 \mu^2\mu^3 \ket{0}_{(NS,NS)}$, and the fundamental vector representation $\mathbf{10}$ of $SO(10)$ formed by the 10 untwisted generators. In addition to these states we have massless $\mu^{a}_{-\frac{1}{6}} \bar z_{-\frac{1}{3}}^b\ket{0}_{(NS,NS)}$, which generate three copies of the fundamental $\textbf{3}$ of $SU(3)$, but are invariant under $r$. And finally there are no other massless states in the remaining sectors. 

Notice that we can combine the fundamental $\mathbf{10}$ of $SO(10)$ with the fundamental $\overline{\mathbf{16}}$ of $SO(10)$ to build a fundamental representation of $E_6$ given by $\overline{\mathbf{27}} \cong \mathbf{10} \oplus \overline{\mathbf{16}}$. A summary of these results is in Table~\ref{tab:twisted}. Finally an identical analysis shows that the $r^2$ twisted sectors simply gives $\overline{\mathbf{3}}$ instead of $\mathbf{3}$.

\begin{table}
    \begin{center}
    \begin{tabular}{ccc}
    &\multicolumn{2}{c}{\textbf{Representation} or \textbf{Generators}}\\
    \bottomrule\midrule
    $r$ & {Left Massless} & Right Massless\\
    \hline
	\multirow{2}{0.5em}{$1$} & $\mathbf{1}\otimes \overline{\mathbf{27}} \otimes \mathbf{1}$ & $\ket{0}_{R}$ \\
	& $\mathbf{3} \otimes \mathbf{1}\otimes \mathbf{1}$ & $\ket{0}_{NS}$ \\
	\hline
    \end{tabular}
    \caption{Collection of massless states in the twisted left and right moving sectors written as either a basis, or a representation of $SO(3)\times E_6\times E_8$ for the right movers or $SO(8)$ for the left as decribed in the text.\label{tab:twisted}}
    \end{center}
\end{table}

\textbf{Degeneracy.} Here is an issue. There is not just one way to twist the fields on the torus $T^6$. So far we examined the twist with boundary conditions $X(\sigma + \pi) = r\cdot X(\sigma)$, but we know that $X \sim X + R \sim X + \omega R$, so we could have equally looked into twisted sectors of the form $X(\sigma + \pi) = r\cdot X(\sigma) + \omega R$ and so on. There are three boundary conditions for each dimension, so we would have $3^3 = 27$ possible twisted sectors for each $r, r^2$. Luckily though since they are all related by shifts, they are all isomorphic. The only tangible impact this has in the final calculation of the massless spectrum is that we need to add each twisted sector $27$ times.

\subsubsection{Orbifold Massless Spectrum}
Finally, the moment we've all been waiting for: combining left and right movers in each sector to figure out the massless content of the orbifolded theory.

From the untwisted sector we have states of the form $\alpha^{\mu}_{-1} \psi^{\nu}_{-\frac{1}{2}} \ket{0} \otimes \ket{0}_{NS}$. These states have $d=4$ spacetime degres of freedom where the symmetric part is the metric, antisymmetric is the axion, and the trace is the dilaton. Then we have states of the form $z^a_{-1} \bar \phi^a_{-\frac{1}{2}} \ket{0}\otimes \ket{0}_{NS}$ and their conjugates which give the internal degrees of freedom for these fields. 

In terms of interesting bosonic content is that for any state $\ket{v} \in \mathbf{334}$ we can write the vector $\psi^{\mu}_{-\frac{1}{2}} \ket{v}\otimes \ket{0}_{NS}$ which is a spacetime gauge boson for $SU(3)\times E_6\times E_8$ since it transforms in its ajdoint! We also have a bunch of other internal bosons that aren't as pretty given by $\bar \phi^a \ket{v}\otimes \ket{0}_{NS}$ with $\ket{v} \in \mathbf{3}\otimes \mathbf{27}\otimes \mathbf{1}$ and their conjugates. Then the fermionic content is contains their superpartners. What is noticeable is the absence of spacetime fermions which is what I promised at the begining of this. 

But fear not! They appear in the twisted sector. In particular there we have states of the form $\ket{v}\otimes \ket{0}_{R}^r$ for $\ket{v} \in \mathbf{1}\otimes \overline{\mathbf{27}} \otimes \mathbf{1}$. If anything we have 27 of them (and their conjugates which would correspond to spacetime fermions with the opposite chirality. What is apparent is that they transform in the fundamental of $E_6$, which means that we have charged chiral spacetime fermions under $E_6$! The full field content is shown in Table~\ref{tab:orbifold}.
\begin{table}
\begin{center}
\begin{tabular}{lllc}
	\textbf{Region} & \textbf{Type} & \textbf{States} & \textbf{Multiplicity} \\
	\bottomrule\toprule
	\multirow{2}{7em}{$d=4$ Spacetime Bosons} 
	& Graviton 	& $\alpha^\mu_{-1} \cdot \psi^{\nu}_{-\frac{1}{2}} \ket{0}\otimes \ket{0}_{NS}$ & 1 \\
	& Axion 	& $\alpha^\mu_{-1} \wedge \psi^{\nu}_{-\frac{1}{2}} \ket{0}\otimes \ket{0}_{NS}$ & 1 \\
	& Dilaton 	& $\text{tr\,}\alpha^\mu_{-1} \psi^{\nu}_{-\frac{1}{2}} \ket{0}\otimes \ket{0}_{NS}$ & 1 \\
	& Gauge 	& $\psi^{\mu}_{-\frac{1}{2}} \ket{v}, \ket{v}\in \mathbf{334}$ & 1\\
	\midrule
	\multirow{1}{7em}{$d=4$ Fermions} 
	& Gravitino & $\alpha^\mu_{-1}\ket{0}\otimes \mathbf{1}$ & 1 \\
	& Dilatino 	& $\alpha^\mu_{-1}\ket{0}\otimes \overline{\mathbf{1}}$ & 1 \\
	\midrule
	\multirow{1}{7em}{Internal Bosons}
	& Graviton 	& $(z^a_{-1}\cdot \bar \phi^{b}_{-\frac{1}{2}} + \bar z^a_{-1}\cdot \phi^{b}_{-\frac{1}{2}}) \ket{0}\otimes \ket{0}_{NS}$ & 1 \\
	& Axion 	& $(z^a_{-1}\wedge \bar \phi^{b}_{-\frac{1}{2}} + \bar z^a_{-1}\wedge \phi^{b}_{-\frac{1}{2}}) \ket{0}\otimes \ket{0}_{NS}$ & 1 \\
	& Dilaton 	& $\text{tr\,}(z^a_{-1} \bar \phi^{b}_{-\frac{1}{2}} + \bar z^a_{-1} \phi^{b}_{-\frac{1}{2}}) \ket{0}\otimes \ket{0}_{NS}$ & 1 \\
	& Scalar 	& $\bar \phi^{a}_{-\frac{1}{2}} \ket{v}\otimes \ket{0}_{NS}, \ket{v} \in \mathbf{3}\otimes \mathbf{27}\otimes \mathbf{1}$ & 1\\
	& Scalar 	& $\phi^{a}_{-\frac{1}{2}} \ket{v}\otimes \ket{0}_{NS}, \ket{v} \in \overline{\mathbf{3}}\otimes \overline{\mathbf{27}}\otimes \mathbf{1}$ & 1\\
	\midrule
	\multirow{2}{7em}{Target Space Fermions}
	& Fermion $\lambda = \pm \frac{1}{2}$ & $(\overline{\mathbf{3}}\otimes \overline{\mathbf{27}} \otimes 1) \otimes \mathbf{3}\oplus (\mathbf{3}\otimes \mathbf{27} \otimes 1) \otimes \overline{\mathbf{3}}$& 1 \\
	& Fermion $\lambda = -\frac{1}{2}$ & $z^a_{-1}\ket{0} \otimes \overline{\mathbf{3}}$ & 1 \\
	& Fermion $\lambda = +\frac{1}{2}$ & $\bar z^a_{-1}\ket{0} \otimes \mathbf{3}$ & 1 \\
	& Chiral Fermion & $\mathbf{1}\otimes \overline{\mathbf{27}} \otimes \mathbf{1} \otimes \ket{0}_R$ & 54 \\
	& Chiral Fermion & $\mathbf{3}\otimes \mathbf{1} \otimes \mathbf{1} \otimes \ket{0}_R$ & 81 \\
	& Chiral Fermion & $\overline{\mathbf{3}}\otimes \mathbf{1} \otimes \mathbf{1} \otimes \ket{0}_R$ & 81 \\
	\midrule
	\multirow{2}{7em}{Target Space Bosons}
	& Scalar & $\mathbf{1}\otimes \overline{\mathbf{27}} \otimes \mathbf{1} \otimes \ket{0}_{NS}$ & 54 \\
	& Scalar & $\mathbf{3}\otimes \mathbf{1} \otimes \mathbf{1} \otimes \ket{0}_{NS}$ & 81 \\
	& Scalar & $\overline{\mathbf{3}}\otimes \mathbf{1} \otimes \mathbf{1} \otimes \ket{0}_{NS}$ & 81 \\
	\midrule
\end{tabular}
\caption{Massless field content of the $E_8\times E_8$ heterotic string $\mathbb{Z}_3$ orbifold. We notice the $N=1$ supersymmetry, as well as the chiral target space fermion states that only have $E_6$  quantum numbers.\label{tab:orbifold}}
\end{center}
\end{table}
\end{document}
