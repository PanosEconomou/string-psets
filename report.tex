\documentclass{homework}
\usepackage{svg}

\title{\textbf{DRAFT} String Theory Final}
\author{Panos Oikonomou}

\begin{document}
\maketitle
\section*{Orbifolds}
How do we make string theory have physics that look like the standard model? We have been using two tools to achieve this: ``editing'' the degrees of freedom of the embedding, and modifying the spacetime the string is embedded in. For example adding fermionic degrees of freedom allowed us to find spacetime Fermions in certain settings. But what really unlocks the door to realistic looking physics is changing the target space in various fun ways.

This is where orbifolds come in. Studying the toolkit of orbifolds will allow us to write down theories in nontrivial target space configurations that lead to interesting physics. In these notes, we begin with an intuitive description of orbifolds through the example of a single compact boson and its $\mathbb{Z}_2$ orbifold, then highlight some generalizations, and finally apply this to the heterotic string compactification that looks a lot like a standard model unification.

\tableofcontents

\section{Orbifolding the Compact Boson}
To study the toolkit of orbifolds we first play with the familiar case of the compact Boson. This is not meant to be an exhaustive review, rather a familiar example which highlights things that we expect on the heterotic string orbifold which have a murkier geometric interpretation. 

We can think of a compact Bosonic field $X : \Sigma \to S^1$ as a way to ``wrap'' the 2-dimensional Riemann surface $\Sigma$ on a circle. This interpretation allows us to steal a core mathematical idea to physics.
\subsection{Mathematical Orbifolds}
Let $G$ be some finite group with an action on a smooth manifold $M$, then the orbifold $M/G$ is the topological space of $G$ equivalence classes on $M$. In other words an orbifold is the shape we obtain by ``folding'' all the points on a manifold that are related by a symmetry on top of each other.

Back to the example of a circle, we have a $\mathbb{Z}_2$ action that maps $X \mapsto -X$, where $X$ is the angle relative to the fixed point of the action as shown in Fig.~\ref{fig:circle-orbifold}. We can then construct the familiar orbifold of the circle which ends up looking like a line segment. The real physics question is, \textit{can we use the field theory that folds $\Sigma$ to $S^1$ to build a theory that folds $\Sigma$ to the orbifold instead}?
\begin{figure}
	\centering
	\includesvg[width=0.7\textwidth]{media/circle_orbifold}
	\caption{The $\mathbb{Z}_2$ orbifold of the circle under the $X\mapsto - X$ map. The fixed point is shown in green.}\label{fig:circle-orbifold}
\end{figure}

\subsection{Untwisted Sector}
To answer this we can cast this orbifold procedure into physics language. We notice that the $\mathbb{Z}_2$ action we described is a global symmetry of the compact free Boson theory, while on the orbifold it is as if this symmetry has become local because the point $X(\sigma, \tau)$ and $-X(\sigma, \tau)$ are the same. So this idea of using the theory of the circle to obtain a theory on the orbifold is identical to gauging a global symmetry. Not only that, but we have already seen examples of this in places like the orientifold construction and the GSO projection. So let's see it in practice.

When we say we want to ``find the theory'' what we really mean is to find some way to describe the Hilbert space. We can start by asking what states of the compact free boson remain in the theory defined on the orbifold. Such states are obtained by acting with the mode operators $\alpha_n$ of a Boson with periodic boundary conditions such that under the transformation $\sigma:\alpha_n\mapsto -\alpha_n$ they remain invariant. This is simply the states with an even number of generators. In fact we can calculate the torus partition function by projecting out the states with a negative $\sigma$ eigenvalue to obtain (as we already have in a homework)
\begin{align*}
	\hat Z 
	&= \text{Tr}_{\mathbb{H}} \frac{1 + \sigma}{2} q^{L_0 - \frac{1}{24}} \bar q^{\tilde L_0 - \frac{1}{24}}
	= \frac{(q \bar q)^{\frac{1}{24}}}{2} \prod_{n \in \mathbb{N}}\abs{1 - q^n}^{-2} + \frac{(q \bar q)^{\frac{1}{24}}}{2} \prod_{n\in \mathbb{N}} \abs{1 + q^{n}}^{-2}\\
	&= \frac{1}{2\sqrt{4\pi^2 \alpha'\text{Im\,}\tau} \abs{\eta(\tau)}^2} + \abs{\frac{\eta(\tau)}{\theta_{10}(\tau)}}.
\end{align*}
Here we recognize the first term as (half) the partition function of the free Boson, while the second is the term that accounts for subtracting the degrees of freedom that are not invariant under $\sigma$.
\subsection{Twisted Sectors}
The partition function we have calculated is not modular invariant because $\theta_{10}(-1/\tau) \propto \theta_{01}(\tau)$. This indicates that the theory must have more states than the ones that we have counted and once we account for them, we will get the extra terms in the torus partition function that will restore modularity. This makes sense because we really only considered ways to wrap $\Sigma$ around a circle such that this was invariant under the orbifold. 

If instead the free boson had the property $X(\sigma + \pi,\tau) = - X(\sigma,\tau)$ while this would not lead to a valid ``wrapping'' around a circle, in the orbifold it's still fine! So let's include states that have these boundary conditions. States such that their holonomy is up to the group action we are gauging by are said to belong to the \textit{twisted sectors} of the theory since we wouldn't find them in the local Hilbert space.

Thankfully in the homework we have also calculated the correct phase that we need to shift $L_0$ by in the twisted sector by calculating the vacuum energy contribution of the twisted boundary conditions. So the trace of the invariant part of the twisted sector is given by
\begin{align*}
	\hat Z_\sigma &=
	\text{Tr}_{\mathbb{H}_\sigma} \frac{1+\sigma}{2} q^{L_0 + \frac{1}{48}} \bar q^{\tilde L_0 + \frac{1}{48}} 
	= (q\bar q)^{\frac{1}{48}} \prod_{n \in \mathbb{N}} \abs{1-q^{n-\frac{1}{2}}}^{-2} (q\bar q)^{\frac{1}{48}} \prod_{n \in \mathbb{N}} \abs{1+q^{n-\frac{1}{2}}}^{-2}\\
	&= \abs{\frac{\eta(\tau)}{\theta_{10}(\tau)}} + \abs{\frac{\eta(\tau)}{\theta_{00}(\tau)}}.
\end{align*} 
Adding these two terms into our partition function from above fixes our modular invariance issue since the first term of the twisted partition function transforms into the second term of the untwisted one, while the last term is modular invariant by itself.
\subsection{Generalizing}
We are by no means done with exploring the $\mathbb{Z}_2$ orbifold of the free boson, but we have encountered enough cool properties to motivate the general construction that we will adopt on the Heterotic string in the next section. 

The first important observation is that the procedure of orbifolding a string theory was to examine how the target space transformations we want to orbifold by form a set of global symmetries for the underlying CFT and gauging that. This is obvious when the fields are simply the coordinates, but as soon as we have fermions in the mix we will see that there might not always be a unique way to interpret which global symmetries correspond to the target space transformations we want to orbifold by.

Another observation is that twisted sectors appear. We used modularity to observe that simply finding the invariant states under the global transformations was not enough to obtain the full spectrum of the orbifolded theory. We resolved this by using our intuition to realize that configurations with twisted boundary conditions on the circle will have untwisted boundary conditions on its the orbifold. This reflects the fact that when we gauge in a general QFT we not only have to path integrate over all configurations of the connection one form, but also have to sum over all principal bundles. This has a physical effect that we see states with different masses than in the previous theory which come from the shift of the vacuum energy in the twisted sector.
\section{Heterotic String Orbifolds}
% The $E_8 \times E_8$ heterotic string has 10 free bosons $X$ with periodic boundary conditions, 32 left moving positive chirality Majorana spinors $\lambda$ half of them with periodic and the other half with antiperiodic boundary conditions, and 10 right moving positive chirality Majorana Fermions and either periodic or antiperiodic boundary conditions.
For more interesting physics let's examine what happens when we orbifold the heterotic string by a discrete Abelian group. The setup is as follows. We compactify the 10 dimensional heterotic string on a 6 dimensional torus $T^6$ formed by a lattice $\Lambda$ that we will describe in a bit. Then we take some symmetries of the torus and form an Abelian group $G$ and orbifold by them (Technically the toroidal compactification is in itself an orbifold, but we can start directly from there). Therefore the final target space looks like
\begin{equation*}
	\mathbb{R}^{3,1}\times (T^6/G)
\end{equation*}
Assume that we have found a rotation of the torus $\sigma$ with index $k \in \mathbb{N}$, i.e.\ such that $\sigma^k = 1$ then we can find out how such a rotation can be interpreted as a symmetry of the fields in the heterotic string and effectively orbifold by $\mathbb{Z}_k \coloneqq \langle \sigma^k\rangle$. Such a rotation will be part of $SO(6)$ and we can always change our coordinates such that the rotation is along the generators of a $U(1)^3 \subset SO(6)$. Let's call these generators $J_a$ for $a\in\{1,2,3\}$ and the corresponding angles $\theta_a$, which implies that $\sigma=e^{i\theta \cdot J}$ when it acts on the Bosons.

In these coordinates we have effectively split the last 6 toroidal coordinates into three pairs where a $J_a$ rotates the $a^{\text{th}}$ pair. We can define $Z^a = \frac{1}{\sqrt{2}} \left(X^{a_1} + i X^{a_2}\right)$, where $a_1, a_2$ are the indices corresponding to the coordinates that $J_a$ is rotating. Therefore we can write the embedding coordinates as $X = (X^0,\cdots, X^3, Z^1, \bar Z^1, \cdots, Z^3, \bar Z^3)$. The reason for doing this is that now they transform as complex bosons under $\sigma$ 
\begin{align*}
	\sigma \cdot Z^a = e^{i\theta_a} Z^a.
\end{align*}
This theory doesn't only have bosons though. Thanks to supersymmetry we know that the antichiral fermions transform the same way after we define the Dirac fermions $\tilde \phi^a = \frac{1}{\sqrt{2}}\left( \tilde \psi^{a_1} + i \tilde \psi^{a_2} \right)$
\begin{align*}
	\sigma \cdot \tilde \phi^a = e^{i\theta_a} \tilde \phi^a.
\end{align*}
Yet, this transformation doesn't fix how the supercurrents $\lambda$ transform. We know that thy must transform under $SO(16)\times SO(16)$ since they are the currents that generate it (the remaining currents that form $E_8\times E_8$ are the fermionic vertex operators). So it seems that we have freedom to choose a gauge transformation for the $\lambda$'s. Not that much freedom thought. If we assume that the identity transformation in the target space comes with the identity transformation of the $\lambda$'s, then it should be the case that the action of $r$ on the supercurrents still satisfies $r^k = 1$. 

Choosing the $\lambda$ appropriately, a general transformation of $SO(16)\times SO(16)$ can be written in terms of generators of $U(1)^{16} \subset SO(16)\times SO(16)$ so if we call these generators $S_a$ for $a \in {1,\cdots, 16}$ and the corresponding angles $\beta_a$ we have that $r = e^{i\theta\cdot J + i\beta \cdot S}$. We can do a similar trick and define $\mu^a = \frac{1}{\sqrt{2}}\left( \lambda^{a_1} + i \lambda^{a_2} \right)$ such that $r$ acts as
\begin{align*}
	\sigma \cdot \mu^a = e^{i\beta_a}\mu^a.
\end{align*}
A convenient choice of angle is $\beta^a = \theta^a$ for $a\in \{1,\cdots 4\}$ and zero otherwise. Now we know how to interpret the target space orbifold in terms of a symmetry of the fields in our string theory. 

As a final remark the finite index of the rotation $\sigma$ implies that $\frac{k\theta_a}{2\pi}, \frac{k\beta_a}{2\pi} \in \mathbb{Z}$ as a further constraint.
\subsection{Obstructions to gauging}
One key issue that wasn't evident in the free Boson orbifold is that in general we might not be able to construct a consistent string theory for any orbifold. Phases between the terms in the torus partition function can potentially break modularity. This happens when we can't assign a phase to the path integral over a sector that is independent of large gauge transformations. Such an anomaly is called a gravitational anomaly and we must ensure that whatever we try to gauge doesn't lead to one. Thankfully, it can be shown that, if we try to orbifold an Abelian group, the only kind of anomaly that can break modularity is gravitational.

The partition function for any well defined string theory is manifestly modular invariant in the untwisted sector. Therefore, if we see any anomaly it will appear in the partition function contributions from the twisted sectors.

In the twisted sector the fields have boundary conditions augmented by the symmetry action. In our case here is a summary of how that looks like for some natural number $0<n<k$
\begin{align*}
	Z^a(\sigma + \pi) = e^{i\theta_a}Z^a(\sigma) && \tilde \phi^a(\sigma + \pi) = e^{i\theta_a + i\nu} \tilde \phi^a(\sigma) && \mu^a(\sigma + \pi) = e^{i\beta_a + i\nu} \mu^a(\sigma),
\end{align*}
where $\nu$ is an angle that defines the $R$ or $NS$ boundary condition of the corresponding field. The reason why this is relevant is that the modes vacuum energy will be shifted for each field and it might happen that the leading power of $q$ in the partition function is no longer invariant under $\tau \mapsto \tau + 1$. To show this, we know that the partition function is given by
\begin{align*}
	\text{tr\,} q^{L_0- \frac{c}{24}} \bar q^{\tilde L_0 - \frac{c}{24}}\xRightarrow{\tau \mapsto \tau+1} \text{tr\,} e^{2\pi i (L_0 - \tilde L_0)}q^{L_0 - \frac{c}{24}} \bar q^{\tilde L_0- \frac{c}{24}}.
\end{align*}
Therefore what we would like is that first factor to be 1 so that the partition function is invariant, which means that $L_0 - \tilde L_0$ should be an integer over the spectrum.

First in the sector where all the fermions are in the Ramond sector $(\nu =0)$ we use the result we derived in the homework to calculate the level mismatch between the left and right movers. In particular we know that the change in the zero point energy for a complex boson is given by $\Delta E_b = -\frac{\theta}{4\pi}\left( \frac{\theta}{2\pi} - 1 \right)$, and minus that for the fermion. Then we can find that
\begin{align*}
	L_0 - \tilde L_0 \  \text{mod\,}1
	&= - \sum_{a=1}^{3} (N_Z^a - N_{\bar Z}^{a} + \tilde N_Z^a - \tilde N_{\bar Z}^{a} + \tilde N_{\phi}^a - \tilde N_{\bar \phi}^a)\frac{\theta_a}{2\pi} - \sum_{a=1}^{16} (N_{\mu}^a - N_{\bar \mu}^a)\frac{\beta_a}{2\pi}\\
	&\ \ \ \, + \sum_{a=1}^{3} \frac{\theta_a}{4\pi}\left( \frac{\theta_a}{2\pi} - 1 \right) - \sum_{a=1}^{16}\frac{\beta_a}{4\pi}\left( \frac{\beta_a}{2\pi} - 1 \right),
\end{align*}
where $N^a_\phi$ is the number operator for the modes of field $\phi^a$. Notice that the top term is an integer multiple of $1/k$, while the bottom is an integer multiple of $1/2k^2$, therefore if we were to simply include the twisted sectors as they are in the partition function, then it would not be modular invariant. In fact, there is a chance that we pick phases such that the bottom contribution from the vacuum energy (i.e. the conformal weight of the vacuum) never cancels. Therefore we must impose the condition
\begin{align*}
	\sum_{a=1}^{3} \frac{\theta_a}{4\pi}\left( \frac{\theta_a}{2\pi} - 1 \right) - \sum_{a=1}^{16}\frac{\beta_a}{4\pi}\left( \frac{\beta_a}{2\pi} - 1 \right)\  \text{mod\,}1 = \frac{m}{k},
\end{align*}
for some integer $m$ at least, in order to even have a chance that these two would cancel out. This is not enough however, because there is no reason why the oscillators would cancel this by an integer. Thankfully, we are not summing over all possible states, only the ones that are invariant under $\mathbb{Z}_k$. In fact, if we calculate the generators $J_a$ and $S_a$ using the corresponding creation and annihilation operators in the twisted sectors we find that the action of $\sigma$ is exactly the oscillator part in $L_0 - \tilde L_0$, so imposing invariance implies that the oscillator part should cancel. Doing a similar analysis in the sectors where one set of fermions is $NS$ (i.e. $\nu = \pi$) and the rest $R$ we obtain the additional conditions
\begin{align*}
\sum_{a=1}^3 \theta_a, \sum_{a=1}^8 \beta_a, \sum_{a=9}^{16} \beta_a \in 4\pi \mathbb{Z}. 
\end{align*} 
\subsection[The Z3 Orbifold]{The $\mathbb{Z}_3$ Orbifold}
% $\rightarrow$ Describe the lattice that forms the $T^6$ that we will later orbifold, why we pick that lattice using roots of unity, and then briefly mention why we should embed the spin connection in the gauge connection.
We finally have enough to describe the orbifold we want to take without running into pesky anomaly issues. The orbifold in this case is $\mathbb{Z}_3$ on a toroidal compacitication of the Heterotic string. Namely, we will construct the torus in the $Z^a$ coordinates by the identifications $Z^a \sim Z^a + R^a$ and $Z^a \sim Z^a + \omega R^a$, where $\omega$ is the 3rd root of unity, i.e. $\omega^3 = 1$, and $R^a$ are some complex numbers defining the lattice spacing. This compactification is the torus $T^6$. The $\mathbb{Z}_3$ that we will find is generated by the rotation $\sigma$ that acts on the coordinates as follows
\begin{align*}
\sigma \cdot Z^1 = \alpha Z^1 && \sigma \cdot Z^2 = \alpha Z^2 && \sigma \cdot Z^3 = \alpha^{-2} Z^3.
\end{align*} 
Or in the notation we had before, $\theta = \left( \frac{2\pi}{3}, \frac{2\pi}{3}, -\frac{4\pi}{3} \right)$, while on the right moving supercurrents it acts the same way, with it permuting only the first three by $\theta$. This satisfies the constraints we found above, therefore we know that trying to orbifold by this group action will yield a modular invariant theory, which is particularly nice. There are other choices we could make for how the angles $\beta$ act, and these would correspond to coupling the theory to different singular flat connections, but we won't examine those. 

The whole point of doing this is to get some cool physics out of this here is what we will show in the next couple of sections. First we will calculate the massless spectrum of the theory carefully and highlight which representations survive in from the untwisted and twisted sectors. Then we will also justify why this orbifold has an unbroken $N=1$ supersymmetry which is pretty cool from a particle phenomenology standpoint since GUT people believe that this might be the case based on studying the self energy of standard model particles. For our purposes though, it is still cool because we get chiral spacetime fermions for free!
\subsection{Low Energy Spectrum}
% $\rightarrow$ Show the low energy spectrum and how it decomposes under the $\mathbb{Z}_3$ in the untwisted and twisted sectors, then highlight what states would survive the projection. In particular use the $SU(3)\times E_6$ subgroup of $E_8$, after justifying how to find it using Dynkin diagrams, to highlight the representations that appear in the $E6$ GUT. Hopefully mention how the standard model is only visible through gravitational interactions, but I need to understand it first.
Notice that the $\mathbb{Z}_3$ is chosen in a special way so as to act only on the first 3 generators $\mu$ of the gauge currents, so it would be convenient to decompose our states in representations of the largest subgroup that is invariant under that $\mathbb{Z}_3$. In fact, we can work out how $e^{i\beta \cdot S}$ is in the center of a $SU(3) \subset E_8\times E_8$, so what we can do is find the largest subgroup that commutes with that $SU(3)$ in $E_8\times E_8$. This turns out to be $E_6$ as shown in Fig.~\ref{fig:dynkin} using some dynkin diagrams. So if we want to classify our massless represetnations in terms of eigenstates of $\sigma$, we should decompose them under $SU(3)\times E_6\times E_8$.
\begin{figure}
    \centering
    \includesvg[width=0.7\textwidth]{media/E8-dynkin}
    \caption{$E_8$ Dynkin diagram showing how we can find an $SU(3)\times E_6$ subgroup. The boxed areas contain roots with the right lattices to generate the two factors.}
    \label{fig:dynkin}
\end{figure}

The fact that we found $E_6$ is exciting on its own merit since at the next section we will see how studying the kind of $E_6$ representations that survive the orbifold will provide a point of contact with the $E_6$ grand unified theory.

Here is an outline for how to figure out the spectrum. We then split the physical massless states in representations of $SU(3) \times E_6\times E_8$ and label them by eigenvalues of $\sigma$ on the right moving and left moving part of the Hilbert space separately. Then we will put left and right together and find out what states in the untwisted sector remain invariant under $\mathbb{Z}_3$ (and GSO). Then we will repeat for the twisted spaces.

\subsubsection{Untwisted Sector}
Thankfully I don't have to work out the decomposition of the $E_8\times E_8$ adjoint representation $\mathbf{248}\oplus \mathbf{248}$ under $SU(3)\times E_6\times E_8$, because Polchinski has done it for us. Namely, we find
\begin{align*}
\mathbf{248} \cong (\mathbf{8} \otimes \mathbf{1}) \oplus (\mathbf{1}\otimes \mathbf{78}) \oplus (\mathbf{3}\otimes \mathbf{27}) \oplus (\overline{\mathbf{3}} \otimes \overline{\mathbf{27}}),
\end{align*} 
where the notation $a\otimes b$ means that $a$ is a representation of $SU(3)$ and $b$ is a representation of $E_6$. It is worth noting that $\textbf{78}$ is the adjoint representation of $E_6$, $\textbf{8}$ the adjoint representation of $SU(3)$, and $\textbf{3}$ the fundamental representation of $SU(3)$. 

Now that we know how the representations of the gauge excitations will be decomposed, we only need to find out the eigenvalues under $\sigma$ for each of them. Since $\sigma$ is an element of the center of $SU(3)$ we only need to find out how it acts on $\textbf{1}, \textbf{3}, \overline{\textbf{3}}, \textbf{8}$.

Consider any eigenstate $\ket{v}$ of the $U(1)^{16}$ generators $S_a$ that we have used to label the center of $SO(16)\times SO(16)$ and define the eigenvalues $S_a \ket{v} = s_i\ket{v}$. Then, we know $e^{i\beta \cdot S} \ket{v} = e^{i\beta\cdot s} \ket{v}$. We can use this to find out how $\sigma$ acts on each representation of the gauge group. The center of $SU(3)$ is $Z_{SU(3)} =\{1, \omega, \omega^2\}$, so in the fundamental representation $\textbf{3}$, we have that all the states have the same eigenvalues $s_1 = 0, s_2 = 1/3, s_3 = 2/3$ which implies that $\sigma \ket{v} = \omega\ket{v}$. For the anti-fundamental $\overline{\textbf{3}}$ the order of generators is permutted so we get $\sigma \ket{v} = \omega^{-1}\ket{v} = \omega^2 \ket{v}$. Finally in the adjoint representation $\textbf{8}$ the center acts as identity, so we get $\sigma \ket{v} = \ket{v}$.

There are more physical left moving states than the ones created using modes of the $\lambda$s. In particular we have found that any state of the form $\ket{v} = \alpha_{-1}^{\mu} \ket{0}$ satisfies $(L_0 - 1)\ket{v} = 0$. So it would be nice to see how these decompose into eigenstates of $\sigma$. Here though, things are easier since $\sigma$ doesn act when $\mu \in \{2,3\}$, so such states satsify $\sigma \ket{v} = \ket{v}$. On the other hand, modes of $Z^a$ look like $z^a\coloneqq\frac{1}{\sqrt{2}}\left(\alpha_{k}^{a_1} + i \alpha_k^{a_2}\right)$. Since $\sigma Z^a = \omega Z^a$ we find that states of the form $\ket{v} = z^a_{-1} \ket{0}$ mush satisfy $\sigma \ket{v} = \omega \ket{v}$. Similarly, states of the form $\ket{v} = \bar z^a_{-1} \ket{0}$ mush satisfy $\sigma \ket{v} = \omega^2 \ket{v}$.

Moving on to the right moving sector the bosonic degrees of freedom act in the same way but with $\sim$ on top, ans by supersymmetry so do the fermions in the $NS$ sector. In the Ramond sector, however, we have to play a similar game with the degenerate vacua as we did with the representations of the gauge group. We know that the group is $SO(8)$, but $\sigma$ acts nontrivially on 6 out of the 8 $\tilde \psi^\mu$, so we can decompose the representations of $SO(8)$ in terms of a representation of $SU(3)$ and the chirality of the remaining spinor. 

$\rightarrow$ Insert a nice table summarizing the above summary + use GSO. 

\subsubsection{Twisted Sectors}
The task of classifying the spectrum in the twisted sectors is, at first, daunting, because we find that there are $3^3 = 27$ ways to impose $\sigma$ twisted boundary conditions in $T^6$ and just as many to do it for $\sigma^2$. Turns out the spectra are all isomorphic, so it suffices to do the analysis only one one sector. Here is how to do this computation: In the R-sector use the fact that modes are shifted to account for the $\sigma$-twist, then find the vertex operators corresponding to the Ramond vacuua in order to extract their spinor helicities. Then use them to check which states survive GSO in the twisted sector. Repeat the process for the NS sector. For the left movers we can do the same analysis treating the supercurrents as 36 fermions. Once done we combine to find what states are left.
\subsection{Spacetime Supersymmetry}
$\rightarrow$ I'll explain why we expect that only one spacetime $d=4$ supersymmetry should be present and be spontaneously broken in the weak scale, and use this to motivate the oribifold that we took by highligting cool observations. Perhaps also mention that while the $\textbf{27}$ and $\overline{\textbf{27}}$ that survived the orbifold are only accessible through coupling with gravity.
\end{document}
