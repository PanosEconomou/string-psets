\documentclass{homework}
\usepackage{emoji}

\title{Introduction to String Theory H7}
\author{Panos Oikonomou}

\begin{document}

\maketitle

\problem[1]
Following what we did in class, we can constrain the classical configurations such that
\begin{align*}
X(\sigma + \pi, \tau) &= X(\sigma, \tau) + 2\pi nR\\
X(\sigma, \tau + \beta) &= X(\sigma, \tau) + 2\pi mR
\end{align*} 
for $n,m \in \mathbb{Z}$. This implies that the classical solutions of the equations of motion are
\[
X_{nm}(\sigma,\tau) = 2 Rn \sigma + \frac{2\pi R}{\beta} m\tau.
\]
So we restrict our path integral to functions of the  form $X = X_{mn} + \phi$, where $\phi$ satisfies the boundary conditions trivially. i.e. $\phi(\sigma + \pi, \tau) = \phi(\sigma, \tau) = \phi(\sigma, \tau + \beta)$. Then we can write the action as 
\[
S(X_{mn} + \phi) = S(X_{mn}) + S(\phi) = \frac{2R^2}{\beta} \left( \beta^2 n^2 + \pi^2 m^2 \right) + S(\phi),
\]
where in the first equality we have used that the cross terms are total derivatives integrated on a closed manifold. To calculate the contribution from $S(\phi)$ we notice that the zero mode of $\phi$ does not appear in $S(\phi) = \partial_\mu \phi \partial^\mu \phi$, so while the rest of the modes of $\phi$ are unaware of the periodicity, the zero mode is not. Writing $\phi = \phi(0) + \psi$ we see that
\[
\int[d\phi] e^{-S(\phi)} = \left(\int_{0}^{2\pi R} d\phi(0)\right) \int [d\psi] e^{-S(\psi)} = 2\pi R  \int [d\psi] e^{-S(\psi)}.
\]
Notice that the second factor does not depend on $R$. Therefore we have that 
\[
\text{tr\,}e^{-\beta W(\beta)} = Z_{\psi}(\beta)(2\pi R)\sum_{nm \in \mathbb{Z}} e^{-\frac{2 R^2}{\beta}(\beta^2 n^2 + \pi^2 m^2)}
\]
where $Z_{\phi}(\beta)$ is the $R$-independent partition function of a fixed boson $\psi$ on a torus that we don't have to calculate.
\problem*[2]
We notice that in the compact case, the momentum of the left and right modes is quantized. In fact, we have to copy the free boson Hilbert space for each winding number $n,m$. More precisely, when applied to a state with winding numbers $n,m$ the $L_0$ operators are given by
\begin{align*}
L_0 = \frac{1}{2}\left( Rn + \frac{m}{2R} \right)^2 + N && \bar L_0 = \frac{1}{2}\left( Rn - \frac{m}{2R} \right)^2 + \bar N,
\end{align*} 
where $N = \sum_{n\in\mathbb{N}} \alpha_{-n}\alpha_n$. The partition function at temperature $\beta$ is given by the torus partition function with $\tau = \frac{i\beta}{\pi}$ since the length in $\sigma$ is $\pi$. Therefore we have that on that torus 
\[
e^{-\beta H} = q^{L_0 - \frac{c}{24}} {\bar q}^{\bar L_0 - \frac{c}{24}},
\]
for $c=1$ and $q = e^{2\pi i \tau} = e^{-2\beta}$. Now splitting the $L_0$ into the free boson modes and the multiplicity induced by the winding we can carry out the computation to obtain
\begin{align*}
\text{tr\,}e^{-\beta H} 
&= c(\beta)\sum_{n,m \in \mathbb{Z}} q^{\frac{1}{2}\left( Rn + \frac{m}{2R} \right)^2} {\bar q}^{\frac{1}{2}\left( Rn - \frac{m}{2R} \right)^2}\\
&= c(\beta) \sum_{n,m \in \mathbb{Z}} e^{-2 \beta R^2 n^2 - \frac{\beta}{2 R^2} m^2}.
\end{align*} 
where we have used that
\[
\text{tr\,} q^{N - \frac{1}{24}} {\bar q}^{\bar N - \frac{1}{24}} = c(\beta)
\]
\problem*[3]
To show that the two expressions are the same we need to do something to the sum over $e^{-\pi \beta m^2/2R^2}$ since it does not have the correct form. Using the hint we can introduce the function 
\[
f(\theta) = \sum_{n\in \mathbb{Z}}e^{- \frac{\beta (\theta + 2\pi n)^2}{8\pi^2 R^2}}.
\]
This function is manifestly periodic since $f(\theta + 2\pi k)$ can be obtained by rearranging the sum. In addition we see that $f(0)$ gives the factor we want in $\text{tr\,}e^{-\beta H}$. Using the hint we can express this in terms of its Fourier series like so
\[
f_m = \frac{1}{2\pi} \sum_{n \in \mathbb{Z}}\int_{0}^{2\pi} e^{-\frac{\beta (\theta + 2\pi n)^2}{8\pi ^2 R^2} + i m\theta} d\theta = \frac{1}{2\pi} \int_{\mathbb{R}} e^{-\frac{\beta \theta^2}{8\pi^2 R^2} + im\theta} d\theta = \frac{2\pi R}{\sqrt{2 \pi \beta}} e^{-\frac{2 R^2 \pi^2}{\beta} m^2 }.
\]
So since $f(0) = \sum_{m\in \mathbb{Z}} f_m$ we can plug this in the answer of problem 2 to obtain that
\[
\text{tr\,}e^{-\beta H} = 2\pi R \sum_{n,m\in \mathbb{Z}} e^{-\frac{2 R^2}{\beta}(\beta^2 n^2 + \pi^2 m^2)} C(\beta),
\]
where $C(\beta) =\text{tr\,} q^{N - \frac{1}{24}} {\bar q}^{\bar N - \frac{1}{24}} /\sqrt{2\pi \beta} $ is some factor independent of $R$. Therefore the two ways of calculating the free energy match up to an additive $R$ independent term $\log C(\beta) - \log Z_\psi(\beta)$. The fact that this term must be zero gives a nice way to calculate the path integral $Z_\psi$ though.
\problem*[4]
Let $\mathbb{H}_{a_n}$ be the Fock space generated by the bosonic operator $a_n = \alpha_n/\sqrt{|n|}$.  
\[
\text{tr}_{\mathbb{H}_{a_n}}q^{\alpha_{-n}\alpha_n} = \text{tr}_{\mathbb{H}_{a_n}}q^{n a_{-n}a_n} = \frac{1}{1-q^n}.
\]
Similarly let $\mathbb{H}_{b_n}$ be the Fock space generated by the fermionic operator $b_n$ where $n \in \mathbb{N}-k$ where $k= 0 , \frac{1}{2}$. Then we have that 
\[
\text{tr}_{\mathbb{H}_{b_n}} q^{n b_{-n}b_n} = 1 + q^{n}.
\]
In $D = 10$ the NS and R sectors are given by
\begin{align*}
\mathbb{H}_{NS} = \bigotimes_{i=1}^8\bigotimes_{n\in \mathbb{N}} \mathbb{H}_{a^i_n}\otimes \mathbb{H}_{b^i_{n - \frac{1}{2}}} && \mathbb{H}_{R} = (8_s \oplus 8_c) \otimes \bigotimes_{i=1}^8\bigotimes_{n\in \mathbb{N}} \mathbb{H}_{a^i_n}\otimes \mathbb{H}_{d^i_n},
\end{align*} 
where we have decomposed the Hilbert space formed by the $0$ modes in the Ramond sector in terms of the two chiralities. Restricted to the zero modes of the ramond sector $N_{R}$ is zero and $(-1)^F$ acts by $\pm 1$ based on which chirality we pick. Also, $(-1)^F$ acts as identity in the bosonic part of each Hilbert space. In the NS sector, however, $(-1)^F$ simply counts the number of generators we have attached to the vacuum. Therefore
\[
\text{tr}_{\mathbb{H}_{b_n}} (-1)^Fq^{n b_{-n}b_n} = 1 - q^{n}.
\]
Since the trace of the product is the product of the trace we have that
\begin{align*}
\text{tr}_{NS} q^{N_{NS}} = q^{-\frac{1}{2}} \prod_{n\in \mathbb{N}} \left[\frac{1 + q^{n-\frac{1}{2}}}{1-q^n}\right]^8  && \text{tr}_{NS} (-1)^F q^{N_{NS}} = q^{-\frac{1}{2}} \prod_{n\in \mathbb{N}} \left[\frac{1 - q^{n-\frac{1}{2}}}{1-q^n}\right]^8.
\end{align*} 
In the Ramond sector instead, we notice that the procedure is identical except from the fact that
\begin{align*}
\text{tr}_{8_s \oplus 8_c} q^{N} = \dim 8_s \oplus 8_c = 16 && \text{tr}_{8_s \oplus 8_c} (-1)^Fq^{N} = \text{tr}_{8_s} q^{N} -  \text{tr}_{8_c} q^{N}= 0, 
\end{align*} 
where I have assumed that $8_s$ is the notation for the subspace with positive chirality but at this point the notation is so dense that I might be wrong. Eitherway, the second trace should be zero regardless. This implies that
\begin{align*}
\text{tr}_{R} q^{N_{R}} = 16\prod_{n\in \mathbb{N}} \left[\frac{1 + q^{n}}{1-q^n}\right]^8  && \text{tr}_{R} (-1)^F q^{N_{R}} = 0.
\end{align*} 
\problem*[5]
Assuming that the total number of fermionic multiplicities in the two sectors is the same in the open string at all levels after GSO (which also implies that the bosonic ones are too) implies that  
\[
\frac{1}{2}\text{tr}_{NS} \left[ 1 - (-1)^F \right] q^{N_S} = \frac{1}{2}\text{tr}_{R} \left[ 1 \pm (-1)^F \right] q^{R}.
\]
Writing this out using the previous problem we have that
\[
\frac{1}{2\sqrt{q}} \prod_{n\in \mathbb{N}} \left[\frac{1 + q^{n-\frac{1}{2}}}{1-q^n}\right]^8 - \frac{1}{2\sqrt{q}} \prod_{n\in \mathbb{N}} \left[\frac{1 - q^{n-\frac{1}{2}}}{1-q^n}\right]^8 = 8 \prod_{n\in \mathbb{N}} \left[\frac{1 + q^{n-\frac{1}{2}}}{1-q^n}\right]^8.
\]
We notice that the denominator is the same so we can multiply it to both sides. Then, setting $q = w^{2}$ we recover the identity.
\problem*[6]
We know that $\slashed p \psi = 0$ and $\bar \chi  \slashed p = 0$ because they satisfy Dirac's equation. One useful thing that we need to prove is the following
\[
\Gamma^{\mu_0 \cdots \mu_k} = \Gamma^{\mu_0}\Gamma^{\mu_1 \cdots \mu_k}  + \frac{1}{(k-1)!} \eta^{\mu_0 [\mu_1} \Gamma^{\mu_2 \cdots \mu_k]},
\]
where the second term comes from anticommuting $\Gamma^{\mu_0}$ past the rest of the $\Gamma^\mu$ in the antisymmetrization. We have $k-1$ of them, but it will only have nontrivial anticommutation relations with at most 1 of them. Similarly we can show
\[
\Gamma^{\mu_0 \cdots \mu_k} = \Gamma^{\mu_1 \cdots \mu_k}\Gamma^{\mu_0}  + \frac{(-1)^{k-1}}{(k-1)!} \eta^{\mu_0 [\mu_1} \Gamma^{\mu_2 \cdots \mu_{k}]},
\]
where this time the $(-1)^{k-1}$ factor came from cyclically permuting the antisymmetrized indices once. Therefore, if $k$ is even we have that $2\Gamma^{\mu_0 \cdots \mu_k} = \Gamma^{\mu_0}\Gamma^{\mu_1 \cdots \mu_k} + \Gamma^{\mu_1 \cdots \mu_k}\Gamma^{\mu_0}$. This implies that for even $k$
\[
p_{\mu_0}\Gamma^{\mu_0 \cdots \mu_k} = \frac{1}{2}\slashed p \Gamma^{\mu_1\cdots \mu_k} + \frac{1}{2} \Gamma^{\mu_1\cdots \mu_k} \slashed p 
\]
Therefore we have that the general case (where $k$ is even otherwise it is trivially zero) satisfies
\begin{align*}
\partial_{\mu_0} F^{\mu_0 \cdots \mu_{k}} = ie^{x\cdot p}\bar \chi p_{\mu_0} \Gamma^{\mu_0 \cdots \mu_{k-1}}\psi = \frac{ie^{x\cdot p}}{2} \bar \chi \left[ \slashed p \Gamma^{\mu_1\cdots \mu_k} + \Gamma^{\mu_1\cdots \mu_k} \slashed p  \right]\psi = 0,
\end{align*} 
which is zero because of Dirac's Equation, this holds for all even $k$ including $k=2$. If we call the corresponding form $F_{k+1}$ then we have shown that $d\ast F_{k+1} = 0$ for even $k$ and in class we have shown how $\ast F_k = F_{10-k}$. Since $\ast \ast F_k \propto F_k$ for any $k$ we see that
\[
d F_3 \propto d\ast F_{7} = 0.
\]
% \begin{align*}
% \partial_{\mu} F^{\mu \nu \rho} = i\bar \chi p_{\mu}\Gamma^{\mu\nu \rho } \psi e^{ix\cdot p} =\frac{i}{6}\bar \chi \left[\Gamma^{\rho}\slashed p\Gamma^{\nu}  - \Gamma^{\nu} \slashed p \Gamma^{\rho}\right]\psi e^{ix\cdot p} = \frac{i}{6}\bar \chi \left[\Gamma^{\rho} p^{\nu}  - p^\nu \Gamma^{\rho}\right]\psi e^{ix\cdot p} =0,
% \end{align*} 
% where we didn't write terms that involved $\slashed p \psi, \bar\chi \slashed p$. 
% \[
% p_{\mu} \Gamma^{\mu\nu\rho} = \slashed p \wedge \Gamma^\nu \wedge \Gamma^\rho 
% \]
% Notice that $2\Gamma^{\mu\nu\rho} = \Gamma^\mu \Gamma^{\nu\rho} + \Gamma^{\nu\rho} \Gamma^\mu$. Therefore we have that
% \begin{align*}
% \partial_{\mu} F^{\mu \nu \rho} = i p_\mu F^{\mu \nu \rho} = i\bar \chi \Gamma^{\nu \rho \mu} p_\mu \psi e^{ix\cdot p} = \frac{i}{2} \bar \chi \left[\slashed p \Gamma^{\nu\rho} + \Gamma^{\nu\rho} \slashed p \right]\psi e^{ix\cdot p} = 0.
% \end{align*} 
% which is equal to zero because of Dirac's equation. For the second identity we see that
% \[
% \partial_{[\mu_4}F_{\mu_1\mu_2\mu_3]} = i\bar \chi p_{[\sigma}\Gamma_{\mu\nu\rho]} \psi e^{-ix\cdot p}.
% \]
% We can write $p_{\sigma} = \frac{1}{2}[\slashed p, \Gamma_{\sigma}]_+$ which implies that
% \begin{align*}
% 2p_{[\sigma} \Gamma_{\mu\nu\rho]} = [\slashed p,\Gamma_{[\sigma}]_+ \Gamma_{\mu\nu\rho]} = \slashed p\Gamma_{\sigma\mu\nu\rho} + \Gamma_{[\sigma} \slashed p \Gamma_{\mu\nu\rho]}.
% \end{align*} 
\end{document}