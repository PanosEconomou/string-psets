\documentclass{homework}
\title{Introduction to String Theory H10}
\author{Panos Oikonomou}

\begin{document}
\maketitle
\problem[1]
Given an embedding $X$ we can write the sting Lagrangian (ignoring overall factors and the dillaton) as
\begin{equation*}
	\mathcal{L}(X) = -\ast \text{tr\,} X^\ast G + X^\ast B. 
\end{equation*}
We can perform a $T$-duality on $X^9$ by introducing the lagrange multiplier $\tilde X^9$, the term $2\tilde X^9 dV$ for a one-form $V$, and writing the corresponding pushforward as $\hat X_\ast = X_\ast + V\otimes \partial_9$ where we use the notation $X_\ast = dX^\mu \otimes \partial_\mu$ is the pushforward of the $D-1$ noncompact coordinates. Plugging $\hat X^\ast$ instead of $X^\ast$ in the previous Lagrangian we get (up to a total derivative)
\begin{align*}
	\mathcal{L}(X,\tilde X^9,V) = -\ast \text{tr\,} X^\ast G + X^\ast B - 2V\wedge \ast\left( \frac{G_{99}}{2} V + X^\ast g + \ast X^\ast b - d\tilde X^9 \right)
\end{align*}
where $g=G(\partial_9,\cdot) = G_{9m}dx^m$, $b = b(\partial_{9}, \cdot) = B_{9m} dx^m$, and the hoge duals are taken in the worldsheet metric. Using $\ast^2 d\sigma^\alpha= d\sigma^\alpha$ in 2D minkowski, the equations of motion of $\tilde X^9$ and $V$ are given by
\begin{align*}
	dV &= 0 \implies V = dX^9\\
	V &= \frac{1}{G_{99}}(\ast d\tilde X^9 - X^\ast g + \ast X^\ast b),
\end{align*}
where we have introduced the variable $X^9$ as the $T$-dual of $\tilde X^9$. These equations boil down to $dX^9 = \ast d\tilde X^9$ we saw when $G=\eta, B=0$. Plugging the second equation of motion back in and tryng to put it in the form of the first euqation we obtain 
\begin{align*}
	X^\ast \tilde G &= X^\ast\left[ G + \frac{1}{G_{99}}\left( b\otimes b - g\otimes g \right)\right]\\
	X^\ast \tilde g &= \frac{1}{G_{99}} X^\ast b & \tilde G_{99} = \frac{1}{G_{99}}
\end{align*}
for any $D-1$ dimensional embedding $X$. With these transformations we find 
\begin{align*}
	X^\ast \tilde h = X^\ast\left[ \tilde G - \frac{\tilde g\otimes \tilde g}{\tilde G_{99}}  \right]
					= X^\ast\left[  G - \frac{1}{G_{99}}\left( b\otimes b - g\otimes g  - b\otimes b\right)\right]
					= X^\ast h.
\end{align*}
Since this is true for any $X$, we find $h = \tilde h$.
\problem*[2]
In the spacetime effective action the dillaton couples as $\sqrt{-G} e^{-2\Phi}$. We also know that $\sqrt{-\tilde G}e^{-2\tilde \Phi} = \sqrt{-G} e^{-2\Phi}$. Using the previous problem, we know that 
\begin{equation*}
	G = \begin{pmatrix} h + \frac{gg^T}{G_{99}} & g \\ g^T & G_{99} \end{pmatrix} = \begin{pmatrix} 1 & 0 \\ g^T/G_{99} & 1  \end{pmatrix} \begin{pmatrix} h &0 \\ 0 & G_{99} \end{pmatrix} \begin{pmatrix} 1 & g/G_{99}\\ 0 & 1 \end{pmatrix} \implies \text{det\,}G = G_{99} \text{det\,}h
\end{equation*}
As a result, we have that
\begin{equation*}
	\tilde \Phi = \Phi -\frac{1}{2} \log \sqrt{\frac{\det G}{\det \tilde G}} = \Phi - \frac{1}{2} \log G_{99}.
\end{equation*}
\problem*[3]
The potential for $N$ $D_0$ branes in Type IIA is proprtional to
\begin{equation*}
	V(X) = \int_{dt} \text{tr\,} \left( C\varepsilon_{ijk} X^i [X^j,X^k] + \frac{M_S}{g_S} [X^i, X^j][X^j,X^i]\right).
\end{equation*}
The the stationary points will appear when the variations with respect to $X^n$ vanish.
\begin{align*}
	3C \varepsilon_{nij} [X^i,X^j] + \frac{4 M_S}{g_S} [X^i,[X^i,X^n]] = 0
\end{align*} 
A representation of $SU(2)$ satisfies $[J^i,J^j] = 2\varepsilon_{ijk}J^k$, using the identity $\varepsilon_{nij}\varepsilon_{ijk} = 2 \delta_{nk}$ we can notice that $[J^i, [J^i, J^j]] = 4\epsilon_{nik}\epsilon_{ijk}J^n = -8 J^j$. As a result we can see that if we set $X^n = \sqrt{\frac{8M_S}{3Cg_S}} J^n$ we have that
\begin{equation*}
	3C\varepsilon_{nij}[X^i, X^j] = \frac{4M_S}{g_S} 8J^n = - \frac{4 M_S}{g_S} [X^i,[X^i,X^n]].
\end{equation*}
\problem*[4]
We wrote the free boson in an expansion in terms of $e^{-2in(\tau \pm \sigma)}$ in order for this to have periodicity $e^{i\theta}$ we only need to say that $n \in \mathbb{Z}\mp \frac{\theta}{2\pi}$ which automatically fulfills our periodicity conditions. The rest of the quantization procedure remains the same, apart from the fact that a complex boson is two decoupled real bosons. Let's call the corresponding Heisenberg algerba generators $\alpha_{k}^{\pm}, \beta_{k}^{\pm}$ with the following nozero commutation relations for $k, l\in \mathbb{Z}$
\begin{align*}
	[\alpha_{k}^{\pm},\alpha_{l}^{\pm}] =[\beta_{k}^{\pm},\beta_{l}^{\pm}]=  \left(k \mp \frac{\theta}{2\pi}\right) \delta_{k+l} 
\end{align*}
Since for $\theta \neq 0 \mod 2\pi$ it is not possible for the zero mode (i.e.\ the one proportional to $(\tau - \sigma)$) to yield the twist for $\sigma \mapsto \sigma + \pi$ we will set $\alpha_0 = 0$ (anyway even for the vacuum of the untwisted sector this still holds so it won't affect the calculation).
Since the Hamiltonian for a single oscillator can be written as $(\alpha^\dagger \alpha + \alpha\alpha^\dagger)$ the full Hamiltonian is given by
\begin{equation*}
	H = \sum_{s=\pm 1}\sum_{k \in \mathbb{Z}} \alpha_{-k}^{s}\alpha_{k}^{s} + \beta_{-k}^{s}\beta_{k}^{s}.
\end{equation*}
The vacuum $v$ is the state with highest weight, therefore it is the one such that $\alpha_{k} v = 0$ for $k\geq 0$ (remembering that $k=0$ only exists for $\theta = 0$). Applying the Hamiltonian to the vacuum we obtain
\begin{align*}
	Hv &= \sum_{s=\pm 1} \sum_{k\leq -\frac{1+s}{2}} \left[\alpha_{-k}^s\alpha_{k}^s + \beta_{-k}^s\beta_{k}^s\right]v
	= \sum_{s=\pm 1} \sum_{k\geq \frac{1+s}{2}} 2\left( k - \frac{s\theta}{2\pi} \right) v = \frac{\theta}{\pi}v + 2\sum_{s=\pm 1} \sum_{k>0}\left( k - \frac{s\theta}{2\pi} \right),
\end{align*}
where we've been careful to identify which is the lowest positive energy mode. Thankfully, Polchinski has the correct regularization of the divergent sum above. Using it we obtain
\begin{equation*}
    E(\theta) =  \frac{1}{6} +\frac{\theta}{\pi} - \frac{1}{4}\left( 1 - \frac{\theta}{\pi} \right)^2 - \frac{1}{4}\left( 1 + \frac{\theta}{\pi} \right)^2 = -\frac{1}{3} +\frac{\theta}{\pi}-\frac{\theta^2}{\pi^2}
\end{equation*}
For the anticommuting complex fermion the process is almost the same. The only difference is that instead of using commutation relations when calculating $Hv$ we will use anticommutation relations which will give a negative sign. So for the fermion $E_f(\theta) = - E_b(\theta)$.
\problem*[5]
The mass operator is given by $-P^2$, so we need to calcualte the canonical momentum in the generic Lagrangian or problem 1. The zero mode momentum for each chirality in the presence of constant $B$ can be written as 
\begin{align*}
	p_{L} &= w +\frac{1}{2}G^{-1}\left( k - 2B w \right) = \begin{pmatrix} 2- \frac{1}{2} G^{-1}E & \frac{1}{2}G^{-1}\end{pmatrix} \binom{w}{k}  = \bar E_L \binom{w}{k}\\
	p_{R} &=-w +\frac{1}{2}G^{-1}\left( k - 2B w \right) = \begin{pmatrix} -\frac{1}{2}G^{-1}E & \frac{1}{2}G^{-1}\end{pmatrix} \binom{w}{k} = \bar E_R \binom{w}{k},
\end{align*}
where we defined these last two matrices for future convenience. Therefore the zero mode mass operator is given by
\begin{align*}
	M^2 
    &= 2(Gp_L)^T p_L + 2(Gp_R)^Tp_R = \begin{pmatrix} w & k\end{pmatrix}  \left[ (G\bar E_L)^T \bar E_L + (G\bar E_R)^T \bar E_R\right]\\
    &= 2\begin{pmatrix} w & k\end{pmatrix} \left[\binom{2G- \frac{1}{2}E^T}{ \frac{1}{2}} \begin{pmatrix} 2- \frac{1}{2} G^{-1}E & \frac{1}{2}G^{-1}\end{pmatrix} + \binom{- \frac{1}{2}E^T}{ \frac{1}{2}} \begin{pmatrix} - \frac{1}{2} G^{-1}E & \frac{1}{2}G^{-1}\end{pmatrix}\right]\binom{w}{k}\\
    &= \begin{pmatrix} w & k\end{pmatrix} \begin{pmatrix}
        E^TG^{-1}E & 2 - E^TG^{-1} \\
        2 - G^{-1}E & G^{-1}
    \end{pmatrix}
    \binom{w}{k},
\end{align*}
where we've used that $E^T + E = 4G$ because $B^T = - B$. This is the matrix $\mathcal{G}$ we talked about in class.
\problem*[6]
% The spectrum should be invariant if we flip everything commutativity of addition implies that $M^2$ is invariant. Namely flipping everything involves
% \begin{align*}
% \binom{w}{k} \mapsto L \binom{w}{k} = \binom{k}{w} &&  (G\bar E_{L,R})^T \bar E_{L,R}  \mapsto L (G\bar E_{L,R})^T \bar E_{L,R} L && \text{where } L = \begin{pmatrix} 0 & 1\\ 1 & 0\end{pmatrix}
% \end{align*} 
Notice that since $G$ is restricted to the Eucledian compact coordinates, we can find an invertible $D$ such that $G = D^T D$. I would love to take credit for coming up with this, but this is the only way I could proceed with the hint in lecture which was to say that the big matrix above can be factorized as $F^TF$ for some $F$ given by
\begin{align*}
F = \begin{pmatrix}
    2D & 0 \\
    2D - (D^T)^{-1} E & (D^T)^{-1}
\end{pmatrix}.
\end{align*} 
I think what will happen is that $F' = R F A $ where $R$ is some orthogonal matrix, we could figure this out by writing the metric $G'=\frac{1}{4}(E' + E'^T)$ and carrying out the computation. This would mean that $\mathcal{G}' = A^T F^TFA$ and since the vector $\binom{w}{k}$ transforms by $A^{-1}$ we'd find that the spectrum remains invariant (since the oscillators don't really konw about the transformation that takes $E \to E'$.
% Now we can see that $F'A$ is given by
% \begin{align*}
% F'A = \begin{pmatrix}2D & 0 \\2D - (D^T)^{-1} E' & (D^T)^{-1}\end{pmatrix} \begin{pmatrix} a & b \\ c& d\end{pmatrix}
% \end{align*} 
% We also know that $A^TLA = L$ so we can calculate   
% \begin{align*}
% \bar E'_L &=  \frac{1}{2}G^{-1}\begin{pmatrix}4G - E' & 1\end{pmatrix} = \frac{1}{2}G^{-1}\begin{pmatrix}4G - E' & 1\end{pmatrix}
% \end{align*} 
\end{document}
