\documentclass{homework}
\title{Introduction to String Theory H10}
\author{Panos Oikonomou}

\begin{document}
\maketitle
\problem[1]
Given an embedding $X$ we can write the sting Lagrangian (ignoring overall factors and the dillaton) as
\begin{equation*}
	\mathcal{L}(X) = -\ast \text{tr\,} X^\ast G + X^\ast B. 
\end{equation*}
We can perform a $T$-duality on $X^9$ by introducing the lagrange multiplier $\tilde X^9$, the term $2\tilde X^9 dV$ for a one-form $V$, and writing the corresponding pushforward as $\hat X_\ast = X_\ast + V\otimes \partial_9$ where we use the notation $X_\ast = dX^\mu \otimes \partial_\mu$ is the pushforward of the $D-1$ noncompact coordinates. Plugging $\hat X^\ast$ instead of $X^\ast$ in the previous Lagrangian we get (up to a total derivative)
\begin{align*}
	\mathcal{L}(X,\tilde X^9,V) = -\ast \text{tr\,} X^\ast G + X^\ast B - 2V\wedge \ast\left( \frac{G_{99}}{2} V + X^\ast g + \ast X^\ast b - d\tilde X^9 \right)
\end{align*}
where $g=G(\partial_9,\cdot) = G_{9m}dx^m$, $b = b(\partial_{9}, \cdot) = B_{9m} dx^m$, and the hoge duals are taken in the worldsheet metric. Using $\ast^2 d\sigma^\alpha= d\sigma^\alpha$ in 2D minkowski, the equations of motion of $\tilde X^9$ and $V$ are given by
\begin{align*}
	dV &= 0 \implies V = dX^9\\
	V &= \frac{1}{G_{99}}(\ast d\tilde X^9 - X^\ast g + \ast X^\ast b),
\end{align*}
where we have introduced the variable $X^9$ as the $T$-dual of $\tilde X^9$. These equations boil down to $dX^9 = \ast d\tilde X^9$ we saw when $G=\eta, B=0$. Plugging the second equation of motion back in and tryng to put it in the form of the first euqation we obtain 
\begin{align*}
	X^\ast \tilde G &= X^\ast\left[ G + \frac{1}{G_{99}}\left( b\otimes b - g\otimes g \right)\right]\\
	X^\ast \tilde g &= \frac{1}{G_{99}} X^\ast b\\
	\tilde G_{99} &= \frac{1}{G_{99}}
\end{align*}
for any $D-1$ dimensional embedding $X$. With these transformations we find 
\begin{align*}
	X^\ast \tilde h = X^\ast\left[ \tilde G - \frac{\tilde g\otimes \tilde g}{\tilde G_{99}}  \right]
					= X^\ast\left[  G - \frac{1}{G_{99}}\left( b\otimes b - g\otimes g  - b\otimes b\right)\right]
					= X^\ast h.
\end{align*}
Since this is true for any $X$, we find $h = \tilde h$.
\problem*[2]
In the spacetime effective action the dillaton couples as $\sqrt{-G} e^{-2\Phi}$. We also know that $\sqrt{-\tilde G}e^{-2\tilde \Phi} = \sqrt{-G} e^{-2\Phi}$. Using the previous problem, we know that 
\begin{equation*}
	G = \begin{pmatrix} h + \frac{gg^T}{G_{99}} & g \\ g^T & G_{99} \end{pmatrix} = \begin{pmatrix} 1 & 0 \\ g^T/G_{99} & 1  \end{pmatrix} \begin{pmatrix} h &0 \\ 0 & G_{99} \end{pmatrix} \begin{pmatrix} 1 & g/G_{99}\\ 0 & 1 \end{pmatrix} \implies \text{det\,}G = G_{99} \text{det\,}h
\end{equation*}
As a result, we have that
\begin{equation*}
	\tilde \Phi = \Phi -\frac{1}{2} \log \sqrt{\frac{G}{\tilde G}} = \Phi - \frac{1}{2} \log G_{99}.
\end{equation*}
\problem*[3]
The potential for $N$ $D_0$ branes in Type IIA is proprtional to
\begin{equation*}
	V(X) = \int_{dt} \text{tr\,} \left( C\varepsilon_{ijk} X^i [X^j,X^k] + \frac{M_S}{g_S} [X^i, X^j][X^j,X^i]\right).
\end{equation*}
The the stationary points will appear when the variations with respect to $X^n$ vanish.
\begin{align*}
	3C \varepsilon_{nij} [X^i,X^j] + \frac{4 M_S}{g_S} [X^i,[X^i,X^n]] = 0
\end{align*} 
A representation of $SU(2)$ satisfies $[J^i,J^j] = 2\varepsilon_{ijk}J^k$, using the identity $\varepsilon_{nij}\varepsilon_{ijk} = 2 \delta_{nk}$ we can notice that $[J^i, [J^i, J^j]] = 4\epsilon_{nik}\epsilon_{ijk}J^n = -8 J^j$. As a result we can see that if we set $X^n = \sqrt{\frac{8M_S}{3Cg_S}} J^n$ we have that
\begin{equation*}
	3C\varepsilon_{nij}[X^i, X^j] = \frac{4M_S}{g_S} 8J^n = - \frac{4 M_S}{g_S} [X^i,[X^i,X^n]].
\end{equation*}
\problem*[4]
We wrote the free boson in an expansion in terms of $e^{-2in(\tau \pm \sigma)}$ in order for this to have periodicity $e^{i\theta}$ we only need to say that $n \in \mathbb{Z}\mp \frac{\theta}{2\pi}$ which automatically fulfills our periodicity conditions. The rest of the quantization procedure remains the same, apart from the fact that a complex boson is two decoupled real bosons. Let's call the corresponding Heisenberg algerba generators $\alpha_{k}^{\pm}, \beta_{k}^{\pm}$ with the following nozero commutation relations for $k, l\in \mathbb{Z}$
\begin{align*}
	[\alpha_{k}^{\pm},\alpha_{l}^{\pm}] =[\beta_{k}^{\pm},\beta_{l}^{\pm}]=  \left(k \mp \frac{\theta}{2\pi}\right) \delta_{k+l} 
\end{align*}
Therefore the Hamiltonian is given by
\begin{equation*}
	H = E(0) + \sum_{s=\pm 1}\sum_{k \in \mathbb{Z}} \alpha_{-k}^{s}\alpha_{k}^{s} + \beta_{-k}^{s}\beta_{k}^{s} 
\end{equation*}
where $E(0)$ is a term that comes from the Schwartzian derivative and is the only nonvanishing part in the vacuum energy calculation when $\theta =0$ (since the oscillators annihilate the highest weight state). Let's apply this to the vacuum $v$ to obtain 
\begin{align*}
	[H-E(0)] v = 
\end{align*}
% We can introduce the free field $\psi$ defined as follows to undo the twisted boundary conditions 
% \begin{equation*}
% 	\phi(\sigma) = e^{\frac{i \theta \sigma}{\pi}} \psi(\sigma) \implies \psi(\sigma +\pi) = \psi(\sigma).
% \end{equation*}
% Now $\psi$ looks like a free boson in the presence of a gauge field $A = \frac{i\theta}{\pi} d\sigma$. Notice that we can do a coordinate transformation $\psi = \chi + i\rho$ and write the complex boson as 2 free real bosons. Let's focus on one and use that $d\sigma^{\pm} = d\tau \pm d\sigma$. To write down the Hamiltonian it would be helpful to write the chiral and antichiral components in terms of modes that satisfy $\partial_{\pm} u^{\pm} \mp i\frac{\theta}{2\pi} u^\pm = - 2ik u^{\pm}$ (the factor of $\frac{1}{2}$ came from the change to holomorphic coordinates). This means that $u_k^\pm(\tau \pm \sigma) = \exp{-2i\left(k\mp \frac{\theta}{4\pi}\right)(\tau \pm \sigma)}$ where $k\mp \frac{\theta}{4\pi}$ has to be an integer which implies that $k \in \mathbb{Z} \pm \frac{\theta}{4\pi}$. Doing first quantization the algebra of the creation and annihilation operators corresponding to the coefficients of $u^{\pm}_k$ remain the same, but this time $k$ is not an integer. As a result, we can write the free bosons in our theory as (I've set $2\alpha' = 1$)
% \begin{align*}
% 	\partial_{\pm}\chi_{\pm} = \sum_{k\in \mathbb{Z}\pm \frac{\theta}{4\pi}} \alpha_{k}^{\pm} u_k^{\pm} && 
% 	\partial_{\pm}\rho_{\pm} = \sum_{k\in \mathbb{Z}\pm \frac{\theta}{4\pi}} \beta_{k}^{\pm} u_k^{\pm}
% \end{align*}
% As a result, the Hamiltonian is given by
% \begin{align*}
% 	H = \sum_{}
% \end{align*}
% Look up Polchinski 2.9 p.73
% The kinetic term of a free complex boson is given by $d \bar \phi \wedge \ast d\phi$ so in this case,we can introduce the function the free field $\psi$ defined as follows to undo the twisted boundary conditions 
% \begin{equation*}
% 	\phi(\sigma) = e^{\frac{i \theta \sigma}{\pi}} \psi(\sigma) \implies \psi(\sigma +\pi) = \psi(\sigma).
% \end{equation*}
% Not only that, but the Lagrangian now looks like $d_{A} \bar \psi \wedge \ast d_{A} \psi$ where $d_A = d + \frac{i\theta}{\pi} d\sigma$ which is an exterior covariant derivative with connection $A= \frac{i\theta}{\pi}d\sigma$, therefore the momentum operator will be given by $p_{\theta} = p_0 + \frac{\theta}{\pi}$. Similarly the rest of the oscillators should be of the form
% Now we are ready to calculate the twisted Hamiltonian the same way as for an untwisted boson but with this new momentum.
% \begin{equation*}
% 	H(\theta) = L_0(\theta) + \bar L_0(\theta) + E(0) = \alpha_0(\theta)^2 + E(0) + = \alpha' p_\theta^2 + E(0) + \text{osc},
% \end{equation*}
% where $E(0)$ is the vacuum energy at $\theta = 0$ which is obtained by the Scwartzian derivative of the stress tensor, and $\text{osc}$ is a term that contains number operators for oscillators which will vanish anyway since the ground state for the free boson satisfies $\alpha_n \omega = 0$ for $n\geq0$. Using the same state but in the presence of $A$ we find that $p_{\theta}^2 \omega = \frac{\theta^2}{\pi^2} \omega$. As a result we find
% \begin{equation*}
% 	E(\theta) - E(0) = \frac{\alpha' \theta^2}{\pi^2}.
% \end{equation*}
% For the fermion the process is the same, however 
% \problem*[5]

\end{document}
