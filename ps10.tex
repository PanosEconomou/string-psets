\documentclass{homework}
\title{Introduction to String Theory H10}
\author{Panos Oikonomou}

\begin{document}
\maketitle
\problem[1]
Let $\hat g = G(\hat \partial_9, \cdot) = G_{9m}$
%Assume that $X_\ast = dX^\mu \otimes \partial_\mu$ is the pushforward of the $D-1$ noncompact coordinates, and consider the string Lagrnagian in the presence of the Lagrange multiplier $\tilde X^9$ multiplying the one form $V$ given by (ignoring constants in front and the dillaton)0\
%\begin{align*}
%	\mathcal{L}(X,\tilde X^9,V) =\mathcal{L}(X) - 2 V\wedge \ast X^\ast g - G_{99} V\wedge \ast V + 2V\wedge X^\ast b + 2\tilde X^9 dV,
%\end{align*}
%where $\mathcal{L}(X)$ is the Lagrangian for the first $D-1$ coordinates $g=G(\partial_9,\cdot) = G_{9m}dx^m$, $b = b(\partial_{9}, \cdot) = B_{9m} dx^m$, and the hoge duals are taken in the worldsheet metric. The equations of motion of $\tilde X^9$ and $V$ are given by (using $\ast^2 d\sigma^\alpha= d\sigma^\alpha$ in 2D minkowski)
%\begin{align*}
%	dV &= 0 \implies V = dX^9\\
%	V &= \frac{1}{G_{99}}(\ast d\tilde X^9 - X^\ast g + \ast X^\ast b),
%\end{align*}
%where we have introduced the variable $X^9$ as the $T$-dual of $\tilde X^9$. These equations boil down to $dX^9 = \ast d\tilde X^9$ we saw when $G=\eta, B=0$. Plugging the second equation of motion back in and integrating out $V$ we obtain

\end{document}
