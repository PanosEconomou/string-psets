\documentclass{homework}
\usepackage{emoji}

\title{Introduction to String Theory H6}
\author{Panos Oikonomou}

\begin{document}

\maketitle

\problem[1]
The generators $G_k$ have the (anti)commutation relations 
\begin{align*}
[G_k,G_l]_+ = 2 L_{k+l} + \frac{c}{12}(4k^2 - 1)\delta_{k+l} && [L_k,G_l] = \frac{k - 2l}{2} G_{k+l}.
\end{align*} 
Therefore we find that
\begin{align*}
G_{\frac{3}{2}} \phi &= G_{\frac{3}{2}} G_{-\frac{1}{2}} \chi = 2 L_{1} \chi = 2G_{\frac{1}{2}}^2 \chi =  0\\
G_{\frac{1}{2}} \phi &= G_{\frac{1}{2}} G_{-\frac{1}{2}} \chi = 2L_0 \chi = \left( 2a - 1 \right)\chi\\
(L_0 - a)\phi &= (L_0 - a)G_{-\frac{1}{2}} \chi =  G_{-\frac{1}{2}} \left(L_0 + \frac{1}{2} - a\right)\chi =0.
\end{align*} 
As a result, $\boxed{a = \frac{1}{2}.}$
\problem*[2]
We use the commutation relations above, as well as that $D = \frac{2}{3}c$ to find that
\begin{align*}
G_{\frac{3}{2}} \phi &= [G_{\frac{3}{2}}, G_{-\frac{3}{2}}]_+ \chi + \lambda [G_{\frac{3}{2}},G_{-\frac{1}{2}}]_+L_{-1} \chi - \lambda G_{-\frac{1}{2}} [G_{\frac{3}{2}},L_{-1}]\chi \\ 
&= \left( 2L_0 + D \right) \chi + 2\lambda [L_{1},L_{-1}]\chi\\
&= [(1 + 2\lambda)(2a - 3) +D]\chi\\
G_{\frac{1}{2}} \phi &= [G_{\frac{1}{2}},G_{-\frac{3}{2}}]_+\chi + \lambda [G_{\frac{1}{2}},G_{-\frac{1}{2}}]_+L_{-1} \chi - \lambda G_{-\frac{1}{2}} [G_{\frac{1}{2}},L_{-1}]\chi\\
&= 2L_{-1}\chi +2\lambda L_0 L_{-1}\chi - \lambda G_{-\frac{1}{2}}^2\chi\\
&= (2-\lambda (2a-1) -\lambda)L_{-1}\chi\\
&=(2-\lambda) \chi\\
L_0\phi &= [L_0,G_{-\frac{3}{2}} + \lambda G_{-\frac{1}{2}}^3]\chi + (G_{-\frac{3}{2}} + \lambda G_{-\frac{1}{2}}^3) L_0 \chi\\
&= (G_{-\frac{3}{2}} + \lambda G_{-\frac{1}{2}}^3) \left( L_0 + \frac{3}{2} \right)\chi = a \phi.
\end{align*} 
The third equation is satisfied for any $\lambda, a,D$. The second implies $\boxed{\lambda = 2}$, and, using $a = \frac{1}{2}$, the first implies $\boxed{D=10}$.
\problem*[3]
In the $R$ sector, the physical state conditions in the Ramond sector are given by $F_k \phi=0$ for $k \geq 0$. This, particularly implies that $L_0 \phi =F_{0}^2 \phi = 0$. In this case, it is enough to show that $F_0 \phi = F_1 \phi = 0$ because the rest can be obtained with the anticommutation relations
\begin{align*}
[F_k,F_l]_+ = 2 L_{k+l} + \frac{c}{3}k^2\delta_{k+l} && [L_k,F_l] = \frac{k - 2l}{2} F_{k+l}.
\end{align*} 
We can now plug in to obtain
\begin{align*}
F_0 \psi &= L_0 F_{-1}\chi = F_{-1}(L_0 +1) \chi  = 0\\
F_1 \psi &= F_1F_0 F_{-1} \chi = [F_1,F_0]_+ F_{-1} \chi  - F_{0}[F_1,F_{-1}]_+\chi \\
&= 2[L_1,F_{-1}]\chi + 2F_{-1}L_1\chi - F_0\left(2L_0 - \frac{D}{2}\right)\chi= \left( 5 - \frac{D}{2} \right)F_0\chi,
\end{align*} 
which implies that $\boxed{D=10}$ for the state to be physical.
\problem*[4]
The OPEs for the ghosts of the superconformal symmetry with nonzero singular terms are 
\begin{align*}
c(z)b(w) \sim \frac{1}{z-w} && \gamma(z) \beta(w) \sim \frac{1}{z-w},
\end{align*}
where $\sim$ is equivalence up to regular functions in $z-w$, while the rest of the OPEs nonsingular. Therefore we can argue that the OPEs of the following must be nonsingular
\begin{align*}
\partial \beta(z) \beta(w) \sim 0 && \partial \beta (z) \partial \beta(w) \sim 0 && \partial c(z) c(w) \sim 0 && \partial c(z) \partial c(w) \sim 0.
\end{align*} 
Also we notice that $\beta, \gamma$ commute, while $b,c$ anticommute. With these identities we can calculate the leading singularity in the OPE of $T_F$ by skipping the terms with vanishing leading order like so
\begin{align*}
T_F(z) T_F(w) 
&=-2{:}b\gamma{:}(z) {:}c\partial \beta{:}(w) -2{:}c\partial \beta{:}(z){:} b \gamma{:}(w) \\
&\ \ \ -3 {:}b \gamma{:}(z){:} \beta \partial c{:}(w) - 3 {:}\beta \partial c{:}(w){:} b \gamma{:}(w) + \cdots\\
&= -4 \langle b(z) c(w)\rangle \langle\gamma(z) \partial \beta(w)\rangle - 6 \langle\beta (z) \gamma(w)\rangle \langle b(z) \partial c(w)\rangle + \cdots\\
&= -\frac{10}{(z-w)^{3}} + \cdots, 
\end{align*} 
where the $\cdots$ denote subleading terms. We find that $\boxed{\hat c = -10}$.
\problem*[5]
The Dirac spinor representation for $\mathfrak{so}(8)$ has 8 $\Gamma$ matrices. We want to build the generators of each copy of $\mathfrak{so}(2)$ in the subgroup using these matrices. From Polchinski we know that the generators of $\mathfrak{so}(8)$ satisfy
\[
J^{mn} = -\frac{i}{4}[\Gamma^m, \Gamma^n],
\]
$SO(8) = D_4$ has a 4-dimensional Cartan subalgebra generated by $h_{k} = J^{2k,2k+1}$ with $k\in\{0,1,2,3\}$. Since $SO(2)^4$ is a 4-dimensional commuting subgroup of $SO(8)$, its generators are mapped to the $h_k$. As a result, in the Dirac spinor representation we have that
\[
h_k = -\frac{i}{4} [\Gamma^{2k},\Gamma^{2k+1}].
\]
The weight of a vector $v$ in the representation is given by the eigenvalues of $h_k$ for all $k$. To calcualte them, we use the Polchinski definitions of $\Gamma^{k\pm} = \frac{1}{2}(\Gamma^{2k} \pm i \Gamma^{2k+1})$ to express
\[
h_k = \Gamma^{k+} \Gamma^{k-} - \frac{1}{2}.
\]
The reason for doing so, is that $\Gamma^{k\pm}$ anticommute among $k$ and for a fixed $k$ act as creation and annihilation operators for fermions. If the vacuum vector of the Dirac representation is $\omega$, a basis $\psi_s$ can be obtained by acting with creator operators 
\[
\psi_s = \left[\prod_{k=0}^3 (\Gamma^{k-})^{s_k + \frac{1}{2}} \right]\omega,
\]
where $s_k \in \{\pm \frac{1}{2}\}$. As a result, the weight of the state $\psi_s$ is given by $s$ since $h_k$ is the occupation operator minus a half. So the possible weights are the elements of $\{\pm \frac{1}{2}\}^{4}$. However, we see that these are double the weights that we need ($2\times 2^3$). This is because the Dirac representation is reducible for even dimensions. In this case, the two irreducible representations are obtained by taking the two eigenspaces of the anticommuting Chirality operator $\Gamma = \prod_{i=0}^7 \Gamma^i$. We see that
\[
[\Gamma, \Gamma^{k-}]_+ = \frac{1}{2}[\Gamma,\Gamma^{2k} - i\Gamma^{2k+1}]_+ = 0 \implies \Gamma \psi_s = \left(\prod_{k=0}^3 2^ks_k \right)\psi_s.
\]
Therefore the two spinor representations have sets of weights $W^\pm$ given by
\[
\boxed{W^{\pm} = \left\{s \in \left\{\pm \frac{1}{2}\right\}^{4}\ \middle |\  \prod_{k=0}^3 2^ks_k= \pm 1\right \}.}
\]
\problem*[6]
In homework 4 we have calculated the conformal weight $h_{\alpha_0}$ for the vertex operator $V_{\alpha_0} = {:}e^{i\alpha_0 \phi}{:}$ for a single free boson $\phi_0$. In our new conventions ($Q=0, T_0=\frac{1}{4}T_{\text{H4}}, i\alpha_0 = \alpha_{H4}$) we have that $h_{\alpha_0} = \frac{\alpha_0^2}{2}.$ 

Now we have 4 of them, therefore our stress tensor is $ T = \sum_{k=0}^3 T_k$, where $T_k$ is the stress tensor for $\phi_k$. Yet, we know that the the differnet bosons commute, so the corresponding vertex oeprator for $\alpha = (\alpha_0,\alpha_1,\alpha_2,\alpha_3)$ factors to 
\[
V_\alpha = {:}\prod_{k=0}^3V_{\alpha_k}{:} \implies T(z)V_{\alpha} (w) = \sum_{k=0}^3 T_k(z) V_{\alpha} = \frac{\sum_{k=0}^3\alpha_k^2}{2(z-3)^2} V(w) + \frac{\partial V_{\alpha}(w)}{z-w} \implies h_{\alpha} = \frac{\alpha^2}{2},
\]
where we have used the fact that OPEs between $T_k(z)V_l(w)$ are nonsingular unless $k=l$.

In wither spinor representation where $\alpha \in W^{\pm}$ we have that $\alpha^2 = 4\times \frac{1}{2^2} = 1$. But also, when $\alpha$ is a weight in the vector representation we have that $\alpha^2 = (\pm1)^2$. Therefore,
\[
\boxed{h_{\text{vector}} = h_{\text{spinor}} = \frac{1}{2}.}
\]
\end{document}